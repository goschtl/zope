\chapter{Functions}

The \module{grok} module provides access to commonly used functions in the
Zope Component Architecture.

    \section{\function{grok.getSite}}

        \begin{funcdesc}{grok.getSite}{*args}
        foobar
        \end{funcdesc}

    \section{\function{grok.notify}}

        \begin{funcdesc}{grok.notify}{*args}
        foobar
        \end{funcdesc}

The \module{grok} module provides a number of convenience functions to aid in
common tasks.

    \section{\function{grok.url}}

        \begin{funcdesc}{grok.url}{request, object, \optional{, name}}
        Construct a URL for the given \var{request} and \var{object}.

        \var{name} may be a string that gets appended to the object URL.
        Commonly used to construct an URL to a particular view on the object.

        This function returns the constructed URL as a string.

        An example, that uses grok.url inside a view class:
        \begin{verbatim}
import grok

class SomeView(grok.View):

    def render(self):
        parent_url = grok.url(
            self.request, self.context.__parent__, 'index')
        return '<a href="%s">parent</a>' % parent_url
        \end{verbatim}

        \end{funcdesc}