% Complete documentation on the extended LaTeX markup used for Python
% documentation is available in ``Documenting Python'', which is part
% of the standard documentation for Python.  It may be found online
% at:
%
%     http://www.python.org/doc/current/doc/doc.html

\documentclass{manual}
\RequirePackage[latin9]{inputenc}
\usepackage{graphicx}

\title{grok reference}

% Please at least include a long-lived email address;
% the rest is at your discretion.
\authoraddress{
    The grok team\\
    Email: <grok-dev@zope.org>
}

\date{\today}   % update before release!
                % Use an explicit date so that reformatting
                % doesn't cause a new date to be used.  Setting
                % the date to \today can be used during draft
                % stages to make it easier to handle versions.

\release{unreleased}      % release version; this is used to define the
                          % \version macro

\makeindex          % tell \index to actually write the .idx file

\begin{document}

\maketitle

    \begin{quote}
    ``Grok means to understand so thoroughly that the observer becomes a part
    of the observed --- merge, blend, intermarry, lose identity in group
    experience. It means almost everything that we mean by religion,
    philosophy, and science --- it means as little to us (because we are from
    Earth) as color means to a blind man.'' -- Robert A. Heinlein, Stranger in
    a Strange Land
    \end{quote}

\begin{abstract}
This is the grok reference documentation. It is organized by the Python
artefacts that implement the concepts.

Grok makes Zope 3 concepts more accessible for application developers. This
reference is not intended as introductory material for those concepts. Please
refer to the original Zope 3 documentation and the grok tutorial for
introductory material.
\end{abstract}

\tableofcontents

\chapter{Core}

The \module{grok} module defines a few functions to interact with grok itself.


\section{\function{grok.grok} -- Grok a package or module}

    \begin{funcdesc}{grok}{dotted_name}

    Grokking a package or module activates the contained components (like
    models, views, adapters, templates, etc.) and registers them with Zope 3's
    component architecture.

    The \var{dotted_name} must specify either a Python module or package
    that is available from the current PYTHONPATH.

    Grokking a module:

    \begin{enumerate}

        \item Scan the module for known components: models, adapters,
              utilities, views, traversers, templates and subscribers.

        \item Check whether a directory with file system templates exists
        (``<modulename>_templates''). If it exists, load the file system
        templates into the template registry for this module.

        \item Determine the module context. 

        \item Register all components with the Zope 3 component architecture.

        \item Initialize schemata for registered models

    \end{enumerate}

    Grokking a package:

    \begin{enumerate}
        \item Grok the package as a module.

        \item Check for a static resource directory (``static'') and register
              it if it exists.

        \item Recursively grok all sub-modules and sub-packages.

    \end{enumerate}

    \end{funcdesc}



\chapter{Components}

The \module{grok} module defines a set of components that provide basic Zope 3
functionality in a convenient way.

\section{\class{grok.Adapter}}
\label{grok-adapter}

    Implementation, configuration, and registration of Zope 3 adapters.

    \begin{classdesc*}{grok.Adapter}
        Base class to define an adapter. Adapters are automatically registered
        when a module is grokked.

        \begin{memberdesc}{context}
            The adapted object.
        \end{memberdesc}

        \emph{Directives:}

        \begin{methoddesc}{grok.context}{context_obj}
            required, identifies the required object for the
            adaptation.
        \end{methoddesc}

        \begin{methoddesc}{grok.implements}{interface}
            required, identifies the interface the adapter implements.
        \end{methoddesc}

        \begin{methoddesc}{grok.name}{name}
            optional, identifies the name used for the adapter
            registration. If ommitted, no name will be used.
        \end{methoddesc}

    \end{classdesc*}

    \emph{Example:}
    \begin{verbatim}
class EuropeanToUS(grok.Adapter):
    """A travel-version of a power adapter that adapts european sockets to
    american sockets.

    """
    grok.implements(IUSPowerSocket)
    grok.context(EuropeanPowerSocket)
    grok.name('travel-adapter') # Optional, can be ommitted

    def power(self):
        return self.context.power
    \end{verbatim}

\section{\class{grok.AddForm}}
\label{grok-addform}

    Make \module{zope.formlib} forms accessible in \module{grok}.

    \begin{classdesc*}{grok.AddForm}
      Form for adding content.

        \begin{memberdesc}{template}
            The associated template to render.
        \end{memberdesc}

        \begin{memberdesc}[grok.Fields]{form_fields}
            Fields to render. 
        \end{memberdesc}

        \emph{Directives:}

        \begin{methoddesc}{grok.context}{context_obj}
          required, identifies the required object for the adaptation
          (forms are views and views are adapters).
        \end{methoddesc}


    \end{classdesc*}

    A \class{grok.AddForm} is a \class{grok.Form}. See
    \ref{grok-form} (\class{grok.Form}) for deeper insights.

    
    \emph{Example:}
\begin{verbatim}
import grok
from zope import schema

class Mammoth(grok.Model):
    class fields:
        name = schema.TextLine(title=u"Name")
        size = schema.TextLine(title=u"Size", default=u"Quite normal")
        somethingelse = None

class Add(grok.AddForm):
    grok.context(Mammoth)
\end{verbatim}

\section{\class{grok.Annotation}}

\section{\class{grok.Application}}

\section{\class{grok.Container}}

    Mixin base class to define a container object. The container supports the
    zope.app.container.interfaces.IContainer interface and is implemented using
    a BTree, providing reasonable performance for large object collections.

    Typically used together with \class{grok.Model}.

\section{grok.ClassGrokker}

    Grokker for particular classes in a module.

    Subclasses should have a \var{component_class} available.

    This is a base grokker. It exists in \module{grok.components}
    because it is meant to be subclassed by code that extends grok.
    Thus it is like \class{grok.Model}, \class{grok.View}, etc. in
    that they should not be grokked themselves but subclasses of them.
  

\section{\class{grok.DisplayForm}}
\label{grok-displayform}

    See \ref{grok-form} (\class{grok.Form}) for deeper insights.

\section{\class{grok.EditForm}}
\label{grok-editform}

    \begin{classdesc*}{grok.EditForm}
        Form for editing content.

        \begin{memberdesc}{template}
            The associated page template to render. If no template is
            explicitly set, a default page template will be used.
        \end{memberdesc}

        \begin{memberdesc}[grok.Fields]{form_fields}
            Fields to render. 
        \end{memberdesc}

        \emph{Directives:}

        \begin{methoddesc}{grok.context}{context_obj}
          required, identifies the required object for the adaptation
          (forms are views and views are adapters).
        \end{methoddesc}


    \end{classdesc*}

    See \ref{grok-form} (\class{grok.Form}) for deeper insights.

    \emph{Example:}
\begin{verbatim}
import grok
from zope import schema

class Mammoth(grok.Model):
    class fields:
        name = schema.TextLine(title=u"Name")
        size = schema.TextLine(title=u"Size", default=u"Quite normal")
        somethingelse = None

class Add(grok.EditForm):
    grok.context(Mammoth)
\end{verbatim}

\section{\class{grok.Form}}
\label{grok-form}

    Base class for forms, bringing \class{grok.View} and
    \module{zope.formlib} together.

    \begin{classdesc*}{grok.Form}
        Grok forms are \module{zope.formlib} forms and
        \class{grok.View}s.

        Needed, because \module{zope.formlib}'s
        Forms have \code{update}/\code{render} methods which have
        different meanings than \class{grok.View}'s
        \code{update}/\code{render} methods.  We deal with this issue
        by 'renaming' \module{zope.formlib}'s \function{update()} to
        update_form() and by disallowing subclasses to have custom
        render() methods. 

        Usually, you will only use \class{grok.AddForm}
        (\ref{grok-addform}), \class{grok.EditForm}
        (\ref{grok-editform}) or \class{grokDisplayForm}
        (\ref{grok-displayform}).


        \begin{memberdesc}[grok.Fields]{form_fields}
          It is a \class{grok.Fields} representing the fields to
          render.
        \end{memberdesc}

        \begin{memberdesc}{template}
            The associated template. If no template is explicitly
            given, a default template will be used.
        \end{memberdesc}

        \begin{methoddesc}{update}{}
          Subclasses can override this method just like on regular
          \class{grok.View}s. It will be called before any form
          processing happens.
        \end{methoddesc}

        \begin{methoddesc}{update_form}{}
          Update the form, i.e. process form input using widgets.

          On zope.formlib forms, this is what the update() method is.
          In grok views, the update() method has a different meaning.
          That's why this method is called update_form() in grok
          forms.
        \end{methoddesc}

        \begin{methoddesc}{render}{}
          Render the form, either using the form template or whatever
          the actions returned in form_result.
        \end{methoddesc}

        \emph{Directives:}

        \begin{methoddesc}{grok.context}{context_obj}
            required, identifies the required object for the
            adaptation (forms are adapters).
        \end{methoddesc}

    \end{classdesc*}

    \emph{Examples:}

    \subsection{Create grok.Fields with keyword parameters and schema fields}

    A \class{grok.Fields} can receive keyword parameters with schema
    fields.  These should be avaible in the definition order.

\begin{verbatim}
import grok
from zope import schema

class Mammoth(grok.Model):
    pass

class Edit(grok.EditForm):
    form_fields = grok.Fields(
        a = schema.TextLine(title=u"Alpha"),
        b = schema.TextLine(title=u"Beta"),
        g = schema.TextLine(title=u"Gamma"),
        d = schema.TextLine(title=u"Delta"))
\end{verbatim}
Now we get the fields in right order:
\begin{verbatim}
  >>> grok.grok(__name__)

  >>> from zope import component
  >>> from zope.publisher.browser import TestRequest
  >>> request = TestRequest()
  >>> view = component.getMultiAdapter((Mammoth(), request), name='edit')
  >>> len(view.form_fields)
  4
  >>> [w.__name__ for w in view.form_fields]
  ['a', 'b', 'g', 'd']
\end{verbatim}
    
    \subsection{Forms cannot define a render method}

    \emph{Forms cannot define a render method.} Here we show the case
    where the EditForm has an explicit template associate with it. 

\begin{verbatim}
  import grok

  class Mammoth(grok.Model):
    pass

  class Edit(grok.EditForm):
    # not allowed to have a render method
    def render(self):
        return "this cannot be"

  edit = grok.PageTemplate('Foo!')
\end{verbatim}
Leads to:
\begin{verbatim}
  >>> grok.grok(__name__)
  Traceback (most recent call last):
  ...
  GrokError: It is not allowed to specify a custom 'render' method for
  form <class 'grok.tests.form.norender.Edit'>. Forms either use the default
  template or a custom-supplied one.
\end{verbatim}





\section{\class{grok.GlobalUtility}}

\section{\class{grok.Indexes}}

\section{grok.InstanceGrokker}

    Grokker for particular instances in a module.

    Subclasses should have a \var{component_class} available.

    This is a base grokker. It exists in \module{grok.components}
    because it is meant to be subclassed by code that extends grok.
    Thus it is like \class{grok.Model}, \class{grok.View}, etc. in
    that they should not be grokked themselves but subclasses of them.


\section{\class{grok.JSON}}

\section{\class{grok.LocalUtility}}

\section{\class{grok.Model}}

    Base class to define an application model object.

    Model classes support persistence and containment.

\section{grok.ModuleGrokker}

    Base class to define an application model object.

    Model classes support persistence and containment.

\section{\class{grok.MultiAdapter}}

    Base class to define a multi-adapter. Multi-adapters are automatically
    registered.

    The class-level directive \function{grok.adapts} is used to identify
    the objects that can be adapted.

    The class-level directive \function{grok.implements} is used to identify
    the interface(s) this adapter implements.

    The class-level directive \function{grok.name} is used to register the
    multi-adapter with a name. If ommitted, no name will be used.

\section{grok.PageTemplate}

\section{grok.PageTemplateFile}

\section{\class{grok.Site}}

\section{\class{grok.Traverser}}

\section{\class{grok.Utility}}

    Base class to define a utility. Utilities are automatically registered.

    The class-level directive \function{grok.implements} is used to identify
    the interface this utility implements. Utilities must provide exactly one
    interface.

    The class-level directive \function{grok.name} is used to register the
    utility with a name. If ommitted, no name will be used.

\section{\class{grok.View}}

\section{\class{grok.XMLRPC}}


\chapter{Directives}

The \module{grok} module defines a set of directives that allow you to
configure and register your components. Most directives assume a
default, based on the environment of a module. (For example, a view
will be automatically associated with a model if the association can
be made unambigously.)

If no default can be assumed for a value, grok will explicitly tell
you what is missing and how you can provide a default or explicit
assignment for the value in question.


    \section{\function{grok.AutoFields} -- deduce schema fields automatically}

        \begin{funcdesc}{grok.AutoFields}{class_or_interface}
          A class level directive, which can be used inside
          \class{Form} classes to automatically deduce the form fields
          from the schema of the context \var{class_or_interface}.

          Different to most other directives,
          \function{grok.AutoFields} is used more like a function and
          less like a pure declaration.

          The following example makes use of the
          \function{grok.AutoFields} directive, in that one field is
          omitted from the form before rendering:

          \strong{Example:}

          \begin{verbatim}
import grok
from zope import interface, schema

class IMammoth(interface.Interface):
    name = schema.TextLine(title=u"Name")
    size = schema.TextLine(title=u"Size", default=u"Quite normal")

class Mammoth(grok.Model):
    interface.implements(IMammoth)

class Edit(grok.EditForm):
    grok.context(Mammoth)

    form_fields = grok.AutoFields(Mammoth).omit('size')
          \end{verbatim}

          In this example the \code{size} attribute will not show up
          in the resulting edit view.

          \begin{seealso}
            \class{grok.EditForm}, \class{grok.Fields}
          \end{seealso}

        \end{funcdesc}

    \section{\function{grok.adapts} -- declare that a class adapts
      certain objects}

        \begin{funcdesc}{grok.adapts}{*classes_or_interfaces}
          A class-level directive to declare that a class adapts
          objects of the classes or interfaces given in
          \var{*classes_or_interfaces}.

          This directive accepts several arguments.

          It works much like the \module{zope.component}s
          \function{adapts()}, but you do not have to make a ZCML
          entry to register the adapter.

          \strong{Example:}

          \begin{verbatim}
import grok
from zope import interface, schema
from zope.size.interfaces import ISized

class IMammoth(interface.Interface):
    name = schema.TextLine(title=u"Name")
    size = schema.TextLine(title=u"Size", default=u"Quite normal")

class Mammoth(grok.Model):
    interface.implements(IMammoth)

class MammothSize(object):
    grok.implements(ISized)
    grok.adapts(IMammoth)

    def __init__(self, context):
        self.context = context

    def sizeForSorting(self):
        return ('byte', 1000)

    def sizeForDisplay(self):
        return ('1000 bytes')
          \end{verbatim}

          Having \class{MammothSize} available, you can register it as
          an adapter, without a single line of ZCML:

          \begin{verbatim}
>>> manfred = Mammoth()
>>> from zope.component import provideAdapter
>>> provideAdapter(MammothSize)
>>> from zope.size.interfaces import ISized
>>> size = ISized(manfred)
>>> size.sizeForDisplay()
'1000 bytes'
          \end{verbatim}

          \begin{seealso}
            \class{grok.implements}
          \end{seealso}

        \end{funcdesc}

    \section{\function{grok.baseclass} -- declare a class as base}

        \begin{funcdesc}{grok.baseclass}{}
          A class-level directive without argument to mark something
          as a base class. Base classes are are not grokked.

          Another way to indicate that something is a base class, is
          by postfixing the classname with \code{'Base'}.

          The baseclass mark is not inherited by subclasses, so those
          subclasses will be grokked (except they are explicitly
          declared as baseclasses as well).

          \strong{Example:}

          \begin{verbatim}
import grok

class ModelBase(grok.Model):
    pass

class ViewBase(grok.View):
    def render(self):
        return "hello world"

class AnotherView(grok.View):
    grok.baseclass()

    def render(self):
        return "hello world"

class WorkingView(grok.View):
    pass
          \end{verbatim}

          Using this example, only the \class{WorkingView} will serve
          as a view, while calling the \class{ViewBase} or
          \class{AnotherView} will lead to a
          \exception{ComponentLookupError}.


        \end{funcdesc}

    \section{\function{grok.define_permission} -- define a permission}

        \begin{funcdesc}{grok.define_permission}{name}

          A module-level directive to define a permission with name
          \var{name}. Usually permission names are prefixed by a
          component- or application name and a dot to keep them
          unique.

          Because in Grok by default everything is accessible by
          everybody, it is important to define permissions, which
          restrict access to certain principals or roles.

          \strong{Example:}

          \begin{verbatim}
import grok
grok.define_permission('cave.enter')
          \end{verbatim}

          \begin{seealso}
            \function{grok.require()}, \class{grok.Permission},
            \class{grok.Role}
          \end{seealso}

          \versionchanged[replaced by \class{grok.Permission}]{0.11}
          
        \end{funcdesc}



    \section{\function{grok.Fields}}

        \begin{funcdesc}{grok.Fields}{*arg}
        foobar
        \end{funcdesc}

    \section{\function{grok.implements}}

        \begin{funcdesc}{grok.implements}{*arg}
        foobar
        \end{funcdesc}

    \section{\function{grok.context}}

        \begin{funcdesc}{grok.context}{*arg}
        foobar
        \end{funcdesc}

    \section{\function{grok.global_utility}}

        \begin{funcdesc}{grok.global_utility}{*arg}
        foobar
        \end{funcdesc}

    \section{\function{grok.i18n.registerTranslations} -- register a
      locales directory}

        \begin{funcdesc}{grok.registerTranslations}{directory}
          A module-level directive, which can be used to register
          a locales directory. It is a Python based equivalent to the
          ZCML directive \code{i18n:registerTranslations}.

          If no value is given, the default \code{'locales'} is taken.

          The directory must exist in the path given relative to the
          package wherein the directive was used.

          The directive takes at most one parameter and can be used
          several times per module.

          A \class{GlobalGrokker} is called for every module in a
          grokked package. If a module does not provide a
          \function{registerTranslations} directive, then
          \code{locales} is assumed as default value and tried to be
          registered.

          This means especially, that if you do no explicit call to
          \function{registerTranslations}, but a local directory with
          name \code{locales} and translations exists, then this
          directory will be registered.

          You can avoid this behaviour by declaring a different
          locales directory in every module of your package(s) which
          also might be empty.

          The following example registers a local \code{locales}
          directory as a source for translations:

          \begin{verbatim}
import grok
grok.i18n.registerTranslations()
          \end{verbatim}

          If this code resides in a location \code{/foo/bar.py}, then
          Zope will look for translations in \code{/foo/locales}. The
          example is a short form for the following:

          \begin{verbatim}
import grok
grok.i18n.registerTranslations('locales')
          \end{verbatim}

          You can, given that the given directories exist, declare
          several locale-directories, which will all be parsed:

          \begin{verbatim}
import grok
grok.i18n.registerTranslations('savannah')
grok.i18n.registerTranslations('landofoz')
          \end{verbatim}
         

        \end{funcdesc}

    \section{\function{grok.name}}

        \begin{funcdesc}{grok.name}{*arg}
        foobar
        \end{funcdesc}

        Used to associate a component with a name. Typically this directive is
        optional. The default behaviour when no name is given depends on the
        component.

    \section{\function{grok.local_utility}}

        \begin{funcdesc}{grok.local_utility}{*arg}
        foobar
        \end{funcdesc}

    \section{\function{grok.provides}}

        \begin{funcdesc}{grok.provides}{*arg}
        foobar
        \end{funcdesc}

    \section{\function{grok.resourcedir --- XXX Not implemented yet}}

        \begin{funcdesc}{grok.resourcedir}{*arg}
        foobar
        \end{funcdesc}

        Resource directories are used to embed static resources like HTML-,
        JavaScript-, CSS- and other files in your application.

        XXX insert directive description here (first: define the name, second:
        describe the default behaviour if the directive isn't given)

        A resource directory is created when a package contains a directory
        with the name \file{static}. All files from this directory become
        accessible from a browser under the URL
        \file{http://<servername>/++resource++<packagename>/<filename>}.

        \begin{bf}Example:\end{bf} The package \module{a.b.c} is grokked and
        contains a directory \file{static} which contains the file
        \file{example.css}. The stylesheet will be available via
        \file{http://<servername>/++resource++a.b.c/example.css}.

        \begin{notice}
        A package can never have both a \file{static} directory and a Python
        module with the name \file{static.py} at the same time. grok will
        remind you of this conflict when grokking a package by displaying an
        error message.
        \end{notice}

        \subsection{Linking to resources from templates}

            grok provides a convenient way to calculate the URLs to static
            resource using the keyword \keyword{static} in page templates:

            \begin{verbatim}
<link rel="stylesheet" tal:attributes="href static/example.css" type="text/css">
            \end{verbatim}

            The keyword \keyword{static} will be replaced by the reference to
            the resource directory for the package in which the template was
            registered.

    \section{\function{grok.require}}

        \begin{funcdesc}{grok.require}{*arg}
        foobar
        \end{funcdesc}

    \section{\function{grok.site}}

        \begin{funcdesc}{grok.site}{*arg}
        foobar
        \end{funcdesc}

    \section{\function{grok.template}}

        \begin{funcdesc}{grok.template}{*arg}
        foobar
        \end{funcdesc}

    \section{\function{grok.templatedir}}

        \begin{funcdesc}{grok.templatedir}{*arg}
        foobar
        \end{funcdesc}


\chapter{Decorators}

grok uses a few decorators to register functions or methods for specific
functionality.

    \section{\function{grok.subscribe} -- Register a function as a subscriber
    for an event}


        \begin{funcdesc}{subscribe}{*classes_or_interfaces}

        Declare that the decorated function subscribes to an event or a
        combination of objects and events. (Similar to Zope 3's
        \function{subscriber} decorator.)

        Applicable on module-level for functions. Requires at least one class
        or interface as argument.

        \end{funcdesc}


    \section{grok.action}



\chapter{Functions}

The \module{grok} module provides a number of convenience functions to aid in
common tasks.

  \section{\function{grok.getSite}}

    \begin{funcdesc}{grok.getSite}{}
    Get the current site object.

      \begin{seealso}
      Site objects are instances of \class{grok.Site} and/or
      \class{grok.Application}.
      \end{seealso}

      \begin{seealso}
      \seetitle
      [http://worldcookery.com/WhereToBuy]
      {Web Component Development With Zope 3, second edition}
      {By Philiip von Weitershaussen; Chapter 18 describes the use of Site
      objects.}
      \end{seealso}

    \end{funcdesc}

  \section{\function{grok.notify}}

    \begin{funcdesc}{grok.notify}{event}
    Send \var{event} to event subscribers.

    Example:
\begin{verbatim}
import grok

class Mammoth(object):
    def __init__(self, name):
        self.name = name

manfred = Mammoth('manfred')

grok.notify(grok.ObjectCreatedEvent(manfred))
\end{verbatim}

      \begin{seealso}
      Grok events provide a selection of common event types.
      \end{seealso}

      \begin{seealso}
      \seetitle
      [http://worldcookery.com/WhereToBuy]
      {Web Component Development With Zope 3, second edition}
      {By Philiip von Weitershaussen; Chapter 16 describes the Zope 3 event
      system.}
      \end{seealso}

    \end{funcdesc}

  \section{\function{grok.url}}

    \begin{funcdesc}{grok.url}{request, object, \optional{, name}}
    Construct a URL for the given \var{request} and \var{object}.

    \var{name} may be a string that gets appended to the object URL. Commonly
    used to construct an URL to a particular view on the object.

    This function returns the constructed URL as a string.

      \begin{seealso}
      View classes derived from \class{grok.View} have a similar \method{url}
      method for constructing URLs.
      \end{seealso}
    \end{funcdesc}


\chapter{Events}

grok provides convenient access to a set of often-used events from Zope 3.
Those events include object and containment events. All events are available as
interface and implemented class.

    \section{grok.IObjectCreatedEvent}

    \section{grok.IObjectModifiedEvent}

    \section{grok.IObjectCopiedEvent}

    \section{grok.IObjectAddedEvent}

    \section{grok.IObjectMovedEvent}

    \section{grok.IObjectRemovedEvent}

    \section{grok.IContainerModifiedEvent}



\chapter{Exceptions}

grok tries to inform you about errors early and with as much guidance as
possible. grok can detect some errors already while importing a module, which
will lead to the \class{GrokImportError}.  Other errors require more context
and can only be detected while executing the \function{grok} function.

    \section{\class{grok.GrokImportError} -- errors while importing a module}

    This exception is raised if a grok-specific problem was found while
    importing a module of your application. \class{GrokImportError} means there
    was a problem in how you are using a part of grok. The error message tries
    to be as informative as possible tell you why something went wrong and how
    you can fix it.

    \class{GrokImportError} is a subclass of Python's \class{ImportError}.

    Examples of situations in which a GrokImportError occurs:

    \begin{itemize}
        \item Using a directive in the wrong context (e.g. grok.templatedir on
        class-level instead of module-level.)
        \item Using a decorator with wrong arguments (e.g. grok.subscribe
        without any argument)
        \item \ldots
    \end{itemize}

    \section{\class{grok.GrokError} -- errors while grokking a module}

    This exception is raised if an error occurs while grokking a module.

    Typically a \class{GrokError} will be raised if one of your modules uses a
    feature of grok that requires some sort of unambigous context to establish
    a reasonable default.

    For example, the \class{grok.View} requires exactly one model to be defined
    locally in the module to assume a default module to be associated with.
    Having no model defined, or more than one model, will lead to an error
    because the context is either underspecified or ambigous.

    The error message of a \class{GrokError} will include the reason for the
    error, the place in your code that triggered the error, and a hint, to help
    you fix the error.

    \begin{classdesc}{GrokError}{Exception}
        \begin{memberdesc}{component}
            The component that was grokked and triggered the error.
        \end{memberdesc}
    \end{classdesc}


\end{document}
