\chapter{Core}

The \module{grok} module defines a few functions to interact with grok itself.


\section{\function{grok.grok} -- Grok a package or module}

    \begin{funcdesc}{grok}{dotted_name}

    Grokking a package or module activates the contained components (like
    models, views, adapters, templates, etc.) and registers them with Zope 3's
    component architecture.

    The \var{dotted_name} must specify either a Python module or package
    that is available from the current PYTHONPATH.

    Grokking a module:

    \begin{enumerate}

        \item Scan the module for known components: models, adapters,
              utilities, views, traversers, templates and subscribers.

        \item Check whether a directory with file system templates exists
        (``<modulename>_templates''). If it exists, load the file system
        templates into the template registry for this module.

        \item Determine the module context. 

        \item Register all components with the Zope 3 component architecture.

        \item Initialize schemata for registered models

    \end{enumerate}

    Grokking a package:

    \begin{enumerate}
        \item Grok the package as a module.

        \item Check for a static resource directory (``static'') and register
              it if it exists.

        \item Recursively grok all sub-modules and sub-packages.

    \end{enumerate}

    \end{funcdesc}

