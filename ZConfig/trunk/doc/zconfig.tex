\documentclass{howto}

\title{ZConfig Package Reference}

%\date{\today}
%\release{0.00}

\author{Zope Corporation}
\authoraddress{
    Lafayette Technology Center\\
    513 Prince Edward Street\\
    Fredericksburg, VA 22401\\
    \url{http://www.zope.com/}
}

\begin{document}
\maketitle

\begin{abstract}
\noindent
This document describes the syntax and API used in configuration files
for components of a Zope installation written by Zope Corporation.  This
configuration mechanism is itself configured using a schema specification
written in XML.

\begin{notice}[warning]
  ZConfig have changed a great deal since this document was initially
  written, and parts of this have not yet been updated, though
  portions have been.  Please be patient as the documentation catches
  up.
\end{notice}
\end{abstract}

\tableofcontents


\section{Introduction \label{intro}}

Zope uses a common syntax and API for configuration files designed for
software components written by Zope Corporation.  Third-party software
which is also part of a Zope installation may use a different syntax,
though any software is welcome to use the syntax used by Zope
Corporation.  Any software written in Python is free to use the
\module{ZConfig} software to load such configuration files in order to
ensure compatibility.  This software is covered by the Zope Public
License, version 2.0.

The \module{ZConfig} package has been tested with Python 2.1 and 2.2.
Python 2.0 is not supported.
It only relies on the Python standard library.


\section{Configuration Syntax \label{syntax}}

Like the \ulink{\module{ConfigParser}}
{http://www.python.org/doc/current/lib/module-ConfigParser.html}
format, this format supports key-value pairs arranged in sections.
Unlike the \module{ConfigParser} format, sections are typed and can be
organized hierarchically, and support delegation of value lookup to
other sections.  Additional files may be included if needed.  Though
both formats are substantially line-oriented, this format is more
flexible.

The intent of supporting nested section is to allow setting up the
configurations for loosely-associated components in a container.  For
example, each process running on a host might get its configuration
section from that host's section of a shared configuration file.  Each
section may use the delegation syntax to share a base configuration
with other components of the same type.

The top level of a configuration file consists of a series of
inclusions, key-value pairs, and sections.

Comments can be added on lines by themselves.  A comment has a
\character{\#} as the first non-space character and extends to the end
of the line:

\begin{verbatim}
# This is a comment
\end{verbatim}

An inclusion is expressed like this:

\begin{verbatim}
%include defaults.conf
\end{verbatim}

The resource to be included can be specified by a relative or absolute
URL, resolved relative to the URL of the resource the
\keyword{\%include} directive is located in.


A key-value pair is expressed like this:

\begin{verbatim}
key value
\end{verbatim}

The key may include any non-white characters except for parentheses.
The value contains all the characters between the key and the end of
the line, with surrounding whitespace removed.

Since comments must be on lines by themselves, the \character{\#}
character can be part of a value:

\begin{verbatim}
key value # still part of the value
\end{verbatim}

Sections may be either empty or non-empty.  An empty section may be
used to provide an alias for another section.

A non-empty section starts with a header, contains configuration
data on subsequent lines, and ends with a terminator.

The header for a non-empty section has this form (square brackets
denote optional parts):

\begin{alltt}
<\var{section-type} \optional{\var{name}} \optional{(\var{basename})} >
\end{alltt}

\var{section-type}, \var{name}, and \var{basename} all have the same
syntactic constraints as key names.

The terminator looks like this:

\begin{alltt}
</\var{section-type}>
\end{alltt}

The configuration data in a non-empty section consists of a sequence
of one or more key-value pairs and sections.  For example:

\begin{verbatim}
<my-section>
    key-1 value-1
    key-2 value-2

    <another-section>
        key-3 value-3
    </another-section>
</my-section>
\end{verbatim}

(The indentation is used here for clarity, but is not required for
syntactic correctness.)

If the \var{basename} component is given for a section header
(regardless of the presence of the name component), that section
acquires additional values from another section having \var{basename}
as its \var{name} and an application-supported type.  For example, an
application that supports the types \code{host} and \code{hostclass}
might use configuration like this:

\begin{verbatim}
<hostclass secondary>
    server-type secondary
    port 1234
</hostclass>

<host grendel (secondary)>
    port 2345
</host>
\end{verbatim}

In this application, sections of type \code{host} would be allowed to
acquire configuration data only from the \code{hostclass} type, so the
section named \code{grendel} would only be allowed to to acquire
configuration data from a section with type \code{hostclass} and name
\code{secondary}.  The \code{hostclass} section named \code{secondary}
could in turn acquire additional key-value pairs from some other
section, based on the allowed type relationships of the
\code{hostclass} type.

The header for empty sections is similar to that of non-empty
sections:

\begin{alltt}
<\var{section-type} \optional{\var{name}} \optional{(\var{basename})} />
\end{alltt}


\subsection{Textual Substitution in Values}

\module{ZConfig} provides a limited way to re-use portions of a value
using simple string substitution.  To use this facility, define named
bits of replacement text using the \keyword{\%define} directive, and
reference these texts from values.

The syntax for \keyword{\%define} is:

\begin{alltt}
%define \var{name} \optional{\var{value}}
\end{alltt}

The value of \var{name} must be a sequence of letters, digits, and
underscores, and may not start with a digit; the namespace for these
names is separate from the other namespaces used with
\module{ZConfig}, and is case-insensitive.  If \var{value} is
omitted, it will be the empty string.  If given, there must be
whitespace between \var{name} and \var{value}; \var{value} will not
include any whitespace on either side, just like values from key-value
pairs.

Names must be defined before they are used, and may not be
re-defined.  All names are associated with the source text in which
they are defined, so distinct sources which are referenced using
\keyword{\%include} are not affected by
definitions created by the resource being parsed.

References to defined names from configuration values use the syntax
described for the \refmodule{ZConfig.substitution} module.
Configuration values which include a \character{\$} as part of the
actual value will need to use \code{\$\$} to get a single
\character{\$} in the result.

The values of defined names are not processed in any way, and may not
contain references to named definitions.

For example, the value for the \code{key} will evaluate to
\code{value}:

\begin{verbatim}
%define name value
key $name
\end{verbatim} %$ <-- bow to font-lock


\section{Standard \module{ZConfig} Datatypes\label{standard-datatypes}}

There are a number of data types which can be identified using the
\code{datatype} attribute on \code{key}, \code{multikey},
\code{schema}, \code{section}, and \code{multisection} elements.
Applications may extend the set of datatypes by calling the
\method{register()} method of the data type regsitry being used or by
using Python dotted-names to refer to conversion routines defined in
code.

The following datatypes are provided by the default type registry.

\begin{definitions}
\term{basic-key}
  The default data type for a key in a ZConfig configuration file.
  The result of conversion is always lower-case, and matches the
  regular expression \regexp{[a-z][-._a-z0-9]*}.

\term{boolean}
  Convert a human-friendly string to a boolean value.  The names
  \code{yes}, \code{on}, and \code{true} convert to \constant{True},
  while \code{no}, \code{off}, and \code{false} convert to
  \constant{False}.  Comparisons are case-insensitive.  All other
  input strings are disallowed.

\term{byte-size}
  A specification of a size, with byte multiplier suffixes (for
  example, \samp{128MB}).  Suffixes are case insensitive and may be
  ``KB'', ``MB'', or ``GB''.

\term{constructor}
  Parse value in the form \samp{fn('1', '2', kw1='a', kw2='b')} into a
  3-tuple where the first element is the string \code{'fn'}, the 2nd
  element is the list \code{['1','2']}, and the 3rd element is the
  dictionary \code{\{'kw1': 'a', 'kw2': 'b'\}}.  This is useful when
  representing a Python-style constructor as a value.  Python syntax
  rules are enforced, but only constants are allowed as positional and
  keyword arguments.  The 3-tuple is returned.

\term{existing-dirpath}
  Validates that the directory portion of a pathname exists.  For
  example, if the value provided is \file{/foo/bar}, \file{/foo} must
  be an existing directory.  No conversion is performed.

\term{existing-directory}
  Validates that a directory by the given name exists on 
  the local filesystem.  No conversion is performed. 

\term{existing-file}
  Validates that a file by the given name exists.  No conversion 
  is performed. 

\term{existing-path}
  Validates that a path (file, directory, or symlink) by the
  given name exists on the local filesystem.  No conversion
  is performed.

\term{float}
  A Python float.  \code{Inf}, \code{-Inf}, and \code{NaN} are not
  allowed.

\term{identifier}
  Any valid Python identifier.

\term{inet-address}
  An internet address expressed as a \code{(\var{hostname},
  \var{port})} pair.  If only the port is specified, an empty string
  will be returned for \var{hostname}.  If the port is omitted,
  \code{None} will be returned for \var{port}.

\term{integer}
  Convert a value to an integer.  This will be a Python \class{int} if
  the value is in the range allowed by \class{int}, otherwise a Python
  \class{long} is returned.

\term{ipaddr-or-hostname}
  Validates a valid IP address or hostname.  If the first 
  character is a digit, the value is assumed to be an IP 
  address.  If the first character is not a digit, the value 
  is assumed to be a hostname.  No conversion is performed. 

\term{key-value}
  Parse a value in the form \code{'A B'} into the list \code{['A',
  'B']}.  Returns the list.

\term{locale}
  Any valid locale specifier accepted by the available
  \function{locale.setlocale()} function.  Be aware that only the
  \code{'C'} locale is supported on some platforms.

\term{logging-level}
  A logging level usable by the \module{logging} package.  Valid
  values are the names \code{critical}, \code{fatal}, \code{error},
  \code{warn}, \code{info}, \code{debug}, and \code{all}, as well as
  integers in the range [0..50].  Converted values are always
  expressed as integers.

\term{null}
  No conversion is performed; the value passed in is the value
  returned.  This is the default data type for section values.

\term{port-number}
  Returns a valid port number as an integer.  Validity does not imply
  that any particular use may be made of the port, however.  For
  example, port number lower than 1024 generally cannot be bound by
  non-root users.

\term{socket-address}
  An address for a socket.  The converted value is a pair containing
  the address family (\constant{AF_INET} or \constant{AF_UNIX}) in the
  first part and the specific address in the second part.  If the
  family is \constant{AF_UNIX}, the specific address will be a
  pathname; if the family is \constant{AF_INET}, the second part will
  be the result of the \strong{inet-address} conversion.

\term{string}
  Returns the input value as a string.  If the source is a Unicode
  string, this implies that it will be checked to be simple 7-bit
  \ASCII.  This is the default data type for key values in
  configuration files.

\term{time-interval}
  A specification of a time interval, with multiplier suffixes,
  e.g. 12h.  Suffixes are case insensitive and may be ``s'' (seconds),
  ``m'' (minutes), ``h'' (hours), or ``d'' (days).

\end{definitions}


\section{\module{ZConfig} --- Basic configuration support}

\declaremodule{}{ZConfig}
\modulesynopsis{Configuration package.}

The main \module{ZConfig} package exports these convenience functions:

\begin{funcdesc}{loadConfig}{schema, url}
  Load and return a configuration from a URL or pathname given by
  \var{url}.  \var{url} may be a URL, absolute pathname, or relative
  pathname.  Fragment identifiers are not supported.  \var{schema} is
  a referennce to a schema loaded by \function{loadSchema()} or
  \function{loadSchemaFile()}.
  The return value is a tuple containing the configuration object and
  a composite handler that, when called with a name-to-handler
  mapping, calls all the handlers for the configuration.
  % XXX need to talk about handlers and the schema definitions still!
\end{funcdesc}

\begin{funcdesc}{loadConfigFile}{schema, file\optional{, url}}
  Load and return a configuration from an opened file object.  If
  \var{url} is omitted, one will be computed based on the
  \member{name} attribute of \var{file}, if it exists.  If no URL can
  be determined, all \keyword{\%include} statements in the
  configuration must use absolute URLs.  \var{schema} is a referennce
  to a schema loaded by \function{loadSchema()} or
  \function{loadSchemaFile()}.
  The return value is a tuple containing the configuration object and
  a composite handler that, when called with a name-to-handler
  mapping, calls all the handlers for the configuration.
  % XXX need to talk about handlers and the schema definitions still!
\end{funcdesc}

\begin{funcdesc}{loadSchema}{url}
  Load a schema definition from the URL \var{url}.  The resulting
  schema object can be passed to \function{loadConfig()} or
  \function{loadConfigFile()}.  The schema object may be used as many
  times as needed.
\end{funcdesc}

\begin{funcdesc}{loadSchemaFile}{file\optional{, url}}
  Load a schema definition from the open file object \var{file}.  If
  \var{url} is given and not \code{None}, it should be the URL of
  resource represented by \var{file}.  If \var{url} is ommitted or
  \code{None}, a URL may be computed from the \member{name} attrigbute
  of \var{file}, if present.  The resulting schema object can
  be passed to \function{loadConfig()} or \function{loadConfigFile()}.
  The schema object may be used as many times as needed.
\end{funcdesc}

The following exceptions are defined by this package:

\begin{excdesc}{ConfigurationError}
  Base class for exceptions specific to the \module{ZConfig} package.
  All instances provide a \member{message} attribute that describes
  the specific error.
\end{excdesc}

\begin{excdesc}{ConfigurationSyntaxError}
  Exception raised when a configuration source does not conform to the
  allowed syntax.  In addition to the \member{message} attribute,
  exceptions of this type offer the \member{url} and \member{lineno}
  attributes, which provide the URL and line number at which the error
  was detected.
\end{excdesc}

\begin{excdesc}{ConfigurationTypeError}
\end{excdesc}

\begin{excdesc}{ConfigurationMissingSectionError}
  Raised when a requested named section is not available.
\end{excdesc}

\begin{excdesc}{ConfigurationConflictingSectionError}
  Raised when a request for a section cannot be fulfilled without
  ambiguity.
\end{excdesc}

\begin{excdesc}{SchemaError}
  Raised when a schema contains an error.  This exception type
  provides the attributes \member{url}, \member{lineno}, and
  \member{colno}, which provide the source URL, the line number, and
  the column number at which the error was detected.  These attributes
  may be \code{None} in some cases.
\end{excdesc}

\begin{excdesc}{SubstitutionReplacementError}
  Raised when the source text contains references to names which are
  not defined in \var{mapping}.  The attributes \member{source} and
  \member{name} provide the complete source text and the name
  (converted to lower case) for which no replacement is defined.
\end{excdesc}

\begin{excdesc}{SubstitutionSyntaxError}
  Raised when the source text contains syntactical errors.
\end{excdesc}


\section{\module{ZConfig.Context} --- Application context}

\declaremodule{}{ZConfig.Context}
\modulesynopsis{Application context.}

The \module{ZConfig} package uses the idea of an \dfn{application
context} to consolidate the connections between the different
components of the package.  Most applications should not need to worry
about the application context at all; the \function{loadURL()} function
in the \module{ZConfig} module uses the default context implementation
to glue everything together.

For applications that need to change the way their configuration data
is handled, the best way to do it is to provide an alternate
application context.  The default implementation is designed to be
subclassed, so this should not prove to be difficult.

\begin{classdesc}{Context}{}
  Constructs an instance of the default application context.  This is
  implemented as an object to allow applications to adjust the way
  components are created and how they are knit together.  This
  implementation is designed to be used once and discarded; changing
  this assumption in a subclass would probably lead to a complete
  replacement of the class.
\end{classdesc}

The context object offers two methods that are used to load a
configuration.  Exactly one of these methods should be called, and it
should be called only once:

\begin{methoddesc}{loadURL}{url}
  Load and return a configuration object from a resource.  The
  resource is identified by a URL or path given as \var{url}.
  Fragment identifiers are not supported.
\end{methoddesc}

\begin{methoddesc}{loadFile}{file\optional{, url}}
  Load and return a configuration from an opened file object.
  If \var{url} is omitted, one will be computed based on the
  \member{name} attribute of \var{file}, if it exists.  If no URL can
  be determined, all \keyword{\%include} statements in the configuration
  must use absolute URLs.
\end{methoddesc}

The following methods are defined to be individually overridable by
subclasses; this should suffice for most context specialization.

\begin{methoddesc}{createNestedSection}{parent, type, name, delegatename}
  Create a new section that represents a child of the section given by
  \var{parent}.  \var{type} is the type that should be given to the
  new section and should always be a string.  \var{name} should be the
  name of the section, and should be a string or \code{None}.
  \var{delegatename} should also be a string or \code{None}; if not
  \code{None}, this will be the name of the section eventually passed
  to the \method{setDelegate()} method of the returned section.  The
  returned section should be conform to the interface of the
  \class{Configuration} class (see the \refmodule{ZConfig.Config}
  module's documentation for more information on this interface).
\end{methoddesc}

\begin{methoddesc}{createToplevelSection}{url}
  Create a new section that represents a section loaded and returned
  by the \method{loadURL()} method of the context object.  The returned
  section should be conform to the interface of the
  \class{ImportingConfiguration} class (see the
  \refmodule{ZConfig.Config} module's documentation for more
  information on this interface).  \var{url} is the resource that will
  be loaded into the new section.
  Since the new section represents the top level of an external
  resource, it's \member{type} and \member{name} attributes should be
  \code{None}.
\end{methoddesc}

\begin{methoddesc}{getDelegateType}{type}
  Return the type of sections to which sections of type \var{type} may
  delegate to, or \code{None} if they are not allowed to do so.
\end{methoddesc}

\begin{methoddesc}{parse}{resource, section}
  This method allows subclasses to replace the resource parser.
  \var{resource} is an object that represents a configuration source;
  it has two attributes, \member{file} and \member{url}.  The
  \member{file} attribute is a file object which provides the content
  of the resource, and \member{url} is the URL from which the resource
  is being loaded.  \var{section} is the section object into which the
  contents of the resources should be loaded.  The default
  implementation implements the configuration language described in
  section~\ref{syntax}.  Providing an
  alternate parser is most easily done by overriding this method and
  calling the parser support methods of the context object from the
  new parser, though different strategies are possible.
\end{methoddesc}

The following methods are provided to make it easy for parsers to
support common semantics for the \keyword{\%include} statement, if
taht is defined for the syntax implemented by the alternate parser.

\begin{methoddesc}{includeConfiguration}{parent, url}
\end{methoddesc}

\begin{methoddesc}{startSection}{parent, type, name, delegatename}
\end{methoddesc}

\begin{methoddesc}{endSection}{parent, type, name, delegatename, section}
\end{methoddesc}


\section{\module{ZConfig.Config} --- Section objects}

\declaremodule{}{ZConfig.Config}
\modulesynopsis{Standard section objects.}


The \module{ZConfig.Config} module provides an implementation of the
standard key-value section.

\begin{classdesc}{Configuration}{type, name, url}
  A typed section with an optional name.  The type is given by the
  \var{type} argument, and the URL the configuration is loaded from is
  given by \var{url}.  Both \var{type} and \var{url} must be strings.
  The optional name of the section is given by \var{name}; if there is
  no name, \var{name} should be \code{None}.
\end{classdesc}

\class{Configuration} objects provide the following attributes and
methods to retrieve information from the section:

\begin{memberdesc}[Configuration]{container}
  The containing section of this section, or \code{None}.
\end{memberdesc}

\begin{memberdesc}[Configuration]{delegate}
  The \class{Configuration} object to which lookups are delegated when
  they cannot be satisfied directly.  If there is no such section,
  this will be \code{None}.
\end{memberdesc}

\begin{methoddesc}[Configuration]{get}{key\optional{, default}}
  Returns the value for \var{key} as a string; a value from the
  delegate section is used if needed.  If there is no value for
  \var{key}, returns \var{default}.
\end{methoddesc}

\begin{methoddesc}[Configuration]{getbool}{key\optional{, default}}
  Returns the value for \var{key} as a \class{bool}.  If there is no
  value for \var{key}, returns \var{default}.  Conversions to
  \class{bool} are case-insensitive; the strings \code{true},
  \code{yes}, and \code{on} cause \code{True} to be returned; the
  strings \code{false}, \code{no}, and \code{off} generate
  \code{False}.  All other strings cause \exception{ValueError} to be
  raised.
\end{methoddesc}

\begin{methoddesc}[Configuration]{getfloat}{key\optional{,
        default\optional{, min\optional{, max}}}}
  Return the value for \var{key} as a float.  If there is no
  value for \var{key}, returns \var{default}.  If the value cannot
  be converted to a float, \exception{ValueError} is raised.  If
  \var{min} is given and the value is less than \var{min}, or if
  \var{max} is given and the value is greater than \var{max},
  \exception{ValueError} is raised.  No range checking is performed if
  neither \var{min} nor \var{max} is given.
\end{methoddesc}

\begin{methoddesc}[Configuration]{getint}{key\optional{,
        default\optional{, min\optional{, max}}}}
  Return the value for \var{key} as an integer.  If there is no
  value for \var{key}, returns \var{default}.  If the value cannot
  be converted to an integer, \exception{ValueError} is raised.  If
  \var{min} is given and the value is less than \var{min}, or if
  \var{max} is given and the value is greater than \var{max},
  \exception{ValueError} is raised.  No range checking is performed if
  neither \var{min} nor \var{max} is given.
\end{methoddesc}

\begin{methoddesc}[Configuration]{getlist}{key\optional{, default}}
  Return the value for \var{key}, converted to a list.  List items are
  separated by whitespace.
\end{methoddesc}

\begin{methoddesc}[Configuration]{has_key}{key}
  Return \code{True} if \var{key} has an associated value, otherwise
  returns \code{False}.
\end{methoddesc}

\begin{methoddesc}[Configuration]{items}{}
  Return a list of key-value pairs from this section, including any
  available from the delegate section.
\end{methoddesc}

\begin{methoddesc}[Configuration]{keys}{}
  Return a list of keys from this section, including any available
  from the delegate section.
\end{methoddesc}

\begin{memberdesc}[Configuration]{name}
  The name of this section, or \code{None}.
\end{memberdesc}

\begin{memberdesc}[Configuration]{type}
  The type of this section as a string.
\end{memberdesc}

\begin{memberdesc}[Configuration]{url}
  The URL of the source this section was loaded from.
\end{memberdesc}


The following method is used to modify the values defined in a
section:

\begin{methoddesc}[Configuration]{addValue}{key, value}
  Add the key \var{key} with the value \var{value}.  If there is
  already a value for \var{key}, \exception{ConfigurationError} is
  raised.
\end{methoddesc}


The following methods are used in retrieving and managing sections:

\begin{methoddesc}[Configuration]{addChildSection}{section}
  Add a section that is a child of the current section.
\end{methoddesc}

\begin{methoddesc}[Configuration]{addNamedSection}{section}
  Add a named section to this section's context.  This is only used to
  add sections that are descendents but not children of the current
  section.
\end{methoddesc}

\begin{methoddesc}[Configuration]{getChildSections}{\optional{type}}
  Returns a sequence of all child sections, in the order in which they
  were added.  If \var{type} is omitted or \code{None}, all sections
  are returned; otherwise only sections of the specified type are
  included.  The delegate is never consulted by this method.
\end{methoddesc}

\begin{methoddesc}[Configuration]{getSection}{type\optional{, name}}
  Returns a single typed section.  The type of the retrieved section
  is given by \var{type}.  If \var{name} is given and not \code{None},
  the name of the section must match \var{name}.  If there is no
  section matching in both name and type,
  \exception{ConfigurationMissingSectionError} is
  raised.  If \var{name} is not given or is \code{None}, there must be
  exactly one child section of type \var{type}; that section is
  returned.  If there is more than one section of type \var{type},
  \exception{ConfigurationConflictingSectionError} is raised.  If
  there is no matching section and a delegate is available, it's
  \method{getSection()} method is called to provide the return value,
  otherwise \code{None} is returned.
\end{methoddesc}

Delegation is supported by one additional method:

\begin{methoddesc}[Configuration]{setDelegate}{section}
  Set the delegate section to \var{section} if not already set.  If
  already set, raises \exception{ConfigurationError}.
\end{methoddesc}

This method is called on each section when the configuration is
completely loaded.  This is called for all sections contained within a
section before it is called on the containing section.

\begin{methoddesc}[Configuration]{finish}{}
  Perform any initialization for the section object that needs to
  occur after the content of the section is loaded and delegation
  chains have been established.  (This method may not have been called
  for delegates before being called on the delegating section.)  The
  default implementation does nothing.
\end{methoddesc}


\section{\module{ZConfig.datatypes} --- Default data type registry}

\declaremodule{}{ZConfig.datatypes}
\modulesynopsis{Default implementation of a data type registry}

The \module{ZConfig.datatypes} module provides the implementation of
the default data type registry and all the standard data types
supported by \module{ZConfig}.  A number of convenience classes are
also provided to assist in the creation of additional datatypes.

A \dfn{datatype registry} is an object that provides conversion
functions for data types.  The interface for a registry is fairly
simple.

A \dfn{conversion function} is any callable object that accepts a
single argument and returns a suitable value, or raises an exception
if the input value is not acceptable.  \exception{ValueError} is the
preferred exception for disallowed inputs, but any other exception
will be properly propogated.

\begin{classdesc}{Registry}{\optional{stock}}
  Implementation of a simple type registry.  If given, \var{stock}
  should be a mapping which defines the ``built-in'' data types for
  the registry; if omitted or \code{None}, the standard set of data
  types is used (see section~\ref{standard-datatypes}, ``Standard
  \module{ZConfig} Datatypes'').
\end{classdesc}

\class{Registry} objects have the following methods:

\begin{methoddesc}{get}{name}
  Return the type conversion routine for \var{name}.  If the
  conversion function cannot be found, an (unspecified) exception is
  raised.  If the name is not provided in the stock set of data types
  by this registry and has not otherwise been registered, this method
  uses the \method{search()} method to load the conversion function.
  This is the only method the \module{ZConfig.schema} module requires.
\end{methoddesc}

\begin{methoddesc}{register}{name, conversion}
  Registery the data type name \var{name} to use the conversion
  function \var{conversion}.  If \var{name} is already registered or
  provided as a stock data type, \exception{ValueError} is raised
  (this includes the case when \var{name} was found using the
  \method{search()} method).
\end{methoddesc}

\begin{methoddesc}{search}{name}
  This is a helper method for the default implementation of the
  \method{get()} method.  If \var{name} is a Python dotted-name, this
  method loads the value for the name by dynamically importing the
  containing module and extracting the value of the name.  The name
  must refer to a usable conversion function.
\end{methoddesc}


The following classes are provided to define conversion functions:

\begin{classdesc}{MemoizedConversion}{conversion}
  Simple memoization for potentially expensive conversions.  This
  conversion helper caches each successful conversion for re-use at a
  later time; failed conversions are not cached in any way, since it
  is difficult to raise a meaningful excpetion providing information
  about the specific failure.
\end{classdesc}

\begin{classdesc}{RangeCheckedConversion}{conversion\optional{,
                                          min\optional{, max}}}
  Helper that performs range checks on the result of another
  conversion.  Values passed to instances of this conversion are
  converted using \var{conversion} and then range checked.  \var{min}
  and \var{max}, if given and not \code{None}, are the inclusive
  endpoints of the allowed range.  Values returned by \var{conversion}
  which lay outside the range described by \var{min} and \var{max}
  cause \exception{ValueError} to be raised.
\end{classdesc}

\begin{classdesc}{RegularExpressionConversion}{regex}
  Conversion that checks that the input matches the regular expression
  \var{regex}.  If it matches, returns the input, otherwise raises
  \exception{ValueError}.
\end{classdesc}


\section{\module{ZConfig.loader} --- Resource loading support}

\declaremodule{}{ZConfig.loader}
\modulesynopsis{Support classes for resource loading}

This module provides some helper classes used by the primary APIs
exported by the \module{ZConfig} package.  These classes may be useful
for some applications, especially applications that want to use a
non-default data type registry.

\begin{classdesc}{Resource}{file, url}
  Object that allows an open file object and a URL to be bound
  together to ease handling.  Instances have the attributes
  \member{file} and \member{url} which store the constructor
  arguments.  These objects also have a \method{close()} method which
  will call \method{close()} on \var{file}, then set the \member{file}
  attribute to \code{None} and the \member{closed} to \code{True}.
\end{classdesc}

\begin{classdesc}{BaseLoader}{}
  Base class for loader objects.  This should not be instantiated
  directly, as the \method{loadResource()} method must be overridden
  for the instance to be used via the public API.
\end{classdesc}

\begin{classdesc}{ConfigLoader}{schema}
  Loader for configuration files.  Each configuration file must
  conform to the schema \var{schema}.  The \method{load*()} methods
  return a tuple consisting of the configuration object and a
  composite handler.
\end{classdesc}

\begin{classdesc}{SchemaLoader}{\optional{registry}}
  Loader that loads schema instances.  All schema loaded by a
  \class{SchemaLoader} will use the same data type registry.  If
  \var{registry} is provided and not \code{None}, it will be used,
  otherwise an instance of \class{ZConfig.datatypes.Registry} will be
  used.
\end{classdesc}


\subsection{Loader Objects}

Loader objects provide a general public interface, an inteface which
subclasses must implement, and some utility methods.

The following methods provide the public interface:

\begin{methoddesc}[loader]{loadURL}{url}
  Open and load a resource specified by the URL \var{url}.
  This method uses the \method{loadResource()} method to perform the
  actual load, and returns whatever that method returns.
\end{methoddesc}

\begin{methoddesc}[loader]{loadFile}{file\optional{, url}}
  Load from an open file object, \var{file}.  If given and not
  \code{None}, \var{url} should be the URL of the resource represented
  by \var{file}.  If omitted or \code{None}, the \member{name}
  attribute of \var{file} is used to compute a \code{file:} URL, if
  present.
  This method uses the \method{loadResource()} method to perform the
  actual load, and returns whatever that method returns.
\end{methoddesc}

The following method must be overridden by subclasses:

\begin{methoddesc}[loader]{loadResource}{resource}
  Subclasses of \class{BaseLoader} must implement this method to
  actually load the resource and return the appropriate
  application-level object.
\end{methoddesc}

The following methods can be used as utilities:

\begin{methoddesc}[loader]{normalizeURL}{url-or-path}
  Return a URL for \var{url-or-path}.  If \var{url-or-path} refers to
  an existing file, the corresponding \code{file:} URL is returned.
  Otherwise \var{url-or-path} is checked for sanity: if it
  does not have a schema, \exception{ValueError} is raised, and if it
  does have a fragment identifier, \exception{ConfigurationError} is
  raised.
\end{methoddesc}

\begin{methoddesc}[loader]{openResource}{url}
  Returns a resource object that represents the URL \var{url}.  The
  URL is opened using the \function{urllib2.urlopen()} function, and
  the returned resource object is created using
  \method{createResource()}.
\end{methoddesc}

\begin{methoddesc}[loader]{createResource}{file, url}
  Returns a resource object for an open file and URL, given as
  \var{file} and \var{url}, respectively.  This may be overridden by a
  subclass if an alternate resource implementation is desired.
\end{methoddesc}


\section{\module{ZConfig.substitution} --- String substitution}

\declaremodule{}{ZConfig.substitution}
\modulesynopsis{Shell-style string substitution helper.}

This module provides a basic substitution facility similar to that
found in the Bourne shell (\program{sh} on most \UNIX{} platforms).  

The replacements supported by this module include:

\begin{tableiii}{l|l|c}{code}{Source}{Replacement}{Notes}
  \lineiii{\$\$}{\code{\$}}{(1)}
  \lineiii{\$\var{name}}{The result of looking up \var{name}}{(2)}
  \lineiii{\$\{\var{name}\}}{The result of looking up \var{name}}{}
\end{tableiii}

\noindent
Notes:
\begin{description}
  \item[(1)]  This is different from the Bourne shell, which uses
              \code{\textbackslash\$} to generate a \character{\$} in
              the result text.  This difference avoids having as many
              special characters in the syntax.

  \item[(2)]  Any character which immediately follows \var{name} may
              not be a valid character in a name.
\end{description}

In each case, \var{name} is a non-empty sequence of alphanumeric and
underscore characters not starting with a digit.  If there is not a
replacement for \var{name}, the exception
\exception{SubstitutionReplacementError} is raised.
Note that the lookup is expected to be case-insensitive; this module
will always use a lower-case version of the name to perform the query.

This module provides these functions:

\begin{funcdesc}{substitute}{s, mapping}
  Substitute values from \var{mapping} into \var{s}.  \var{mapping}
  can be a \class{dict} or any type that supports the \method{get()}
  method of the mapping protocol.  Replacement
  values are copied into the result without further interpretation.
  Raises \exception{SubstitutionSyntaxError} if there are malformed
  constructs in \var{s}.
\end{funcdesc}

\begin{funcdesc}{isname}{s}
  Returns \code{True} if \var{s} is a valid name for a substitution
  text, otherwise returns \code{False}.
\end{funcdesc}


\subsection{Examples}

\begin{verbatim}
>>> from ZConfig.substitution import substitute
>>> d = {'name': 'value',
...      'top': '$middle',
...      'middle' : 'bottom'}
>>>
>>> substitute('$name', d)
'value'
>>> substitute('$top', d)
'$middle'
\end{verbatim}


\end{document}
