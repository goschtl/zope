%%%%%%%%%%%%%%%%%%%%%%%%%%%%%%%%%%%%%%%%%%%%%%%%%%%%%%%%%%%%%%%%%%%%%%%%%%%%%%
%
% Copyright (c) 2002, 2003 Zope Corporation and Contributors.
% All Rights Reserved.
%
% This software is subject to the provisions of the Zope Public License,
% Version 2.1 (ZPL).  A copy of the ZPL should accompany this distribution.
% THIS SOFTWARE IS PROVIDED "AS IS" AND ANY AND ALL EXPRESS OR IMPLIED
% WARRANTIES ARE DISCLAIMED, INCLUDING, BUT NOT LIMITED TO, THE IMPLIED
% WARRANTIES OF TITLE, MERCHANTABILITY, AGAINST INFRINGEMENT, AND FITNESS
% FOR A PARTICULAR PURPOSE.
%
%%%%%%%%%%%%%%%%%%%%%%%%%%%%%%%%%%%%%%%%%%%%%%%%%%%%%%%%%%%%%%%%%%%%%%%%%%%%%%

\documentclass{howto}
\usepackage{xmlmarkup}

\newcommand{\datatype}[1]{\strong{#1}}

\title{ZConfig Package Reference}

%\date{27 October 2003}
\release{2.2}
\setshortversion{2.2}

\author{Zope Corporation}
\authoraddress{
    Lafayette Technology Center\\
    513 Prince Edward Street\\
    Fredericksburg, VA 22401\\
    \url{http://www.zope.com/}
}

\begin{document}
\maketitle

\begin{abstract}
\noindent
This document describes the syntax and API used in configuration files
for components of a Zope installation written by Zope Corporation.  This
configuration mechanism is itself configured using a schema specification
written in XML.
\end{abstract}

\tableofcontents


\section{Introduction \label{intro}}

Zope uses a common syntax and API for configuration files designed for
software components written by Zope Corporation.  Third-party software
which is also part of a Zope installation may use a different syntax,
though any software is welcome to use the syntax used by Zope
Corporation.  Any software written in Python is free to use the
\module{ZConfig} software to load such configuration files in order to
ensure compatibility.  This software is covered by the Zope Public
License, version 2.0.

The \module{ZConfig} package has been tested with Python 2.3.  Older
versions of Python are not supported.
\module{ZConfig} only relies on the Python standard library.

Configurations which use \module{ZConfig} are described using
\dfn{schema}.  A schema is a specification for the allowed structure
and content of the configuration.  \module{ZConfig} schema are written
using a small XML-based language.  The schema language allows the
schema author to specify the names of the keys allowed at the top
level and within sections, to define the types of sections which may
be used (and where), the types of each values, whether a key or
section must be specified or is optional, default values for keys, and
whether a value can be given only once or repeatedly.


\section{Configuration Syntax \label{syntax}}

Like the \ulink{\module{ConfigParser}}
{http://docs.python.org/lib/module-ConfigParser.html}
format, this format supports key-value pairs arranged in sections.
Unlike the \module{ConfigParser} format, sections are typed and can be
organized hierarchically.
Additional files may be included if needed.  Schema components not
specified in the application schema can be imported from the
configuration file.  Though both formats are substantially
line-oriented, this format is more flexible.

The intent of supporting nested section is to allow setting up the
configurations for loosely-associated components in a container.  For
example, each process running on a host might get its configuration
section from that host's section of a shared configuration file.

The top level of a configuration file consists of a series of
inclusions, key-value pairs, and sections.

Comments can be added on lines by themselves.  A comment has a
\character{\#} as the first non-space character and extends to the end
of the line:

\begin{verbatim}
# This is a comment
\end{verbatim}

An inclusion is expressed like this:

\begin{verbatim}
%include defaults.conf
\end{verbatim}

The resource to be included can be specified by a relative or absolute
URL, resolved relative to the URL of the resource the
\keyword{\%include} directive is located in.


A key-value pair is expressed like this:

\begin{verbatim}
key value
\end{verbatim}

The key may include any non-white characters except for parentheses.
The value contains all the characters between the key and the end of
the line, with surrounding whitespace removed.

Since comments must be on lines by themselves, the \character{\#}
character can be part of a value:

\begin{verbatim}
key value # still part of the value
\end{verbatim}

Sections may be either empty or non-empty.  An empty section may be
used to provide an alias for another section.

A non-empty section starts with a header, contains configuration
data on subsequent lines, and ends with a terminator.

The header for a non-empty section has this form (square brackets
denote optional parts):

\begin{alltt}
<\var{section-type} \optional{\var{name}} >
\end{alltt}

\var{section-type} and \var{name} all have the same syntactic
constraints as key names.

The terminator looks like this:

\begin{alltt}
</\var{section-type}>
\end{alltt}

The configuration data in a non-empty section consists of a sequence
of one or more key-value pairs and sections.  For example:

\begin{verbatim}
<my-section>
    key-1 value-1
    key-2 value-2

    <another-section>
        key-3 value-3
    </another-section>
</my-section>
\end{verbatim}

(The indentation is used here for clarity, but is not required for
syntactic correctness.)

The header for empty sections is similar to that of non-empty
sections, but there is no terminator:

\begin{alltt}
<\var{section-type} \optional{\var{name}} />
\end{alltt}


\subsection{Extending the Configuration Schema}

As we'll see in section~\ref{writing-schema}, ``Writing Configuration
Schema,'' what can be written in a configuration is controlled by
schemas which can be built from \emph{components}.  These components
can also be used to extend the set of implementations of objects the
application can handle.  What this means when writing a configuration
is that third-party implementations of application object types can be
used wherever those application types are used in the configuration,
if there's a \module{ZConfig} component available for that
implementation.

The configuration file can use an \keyword{\%import} directive to load
a named component:

\begin{verbatim}
%import Products.Ape
\end{verbatim}

The text to the right of the \keyword{\%import} keyword must be the
name of a Python package; the \module{ZConfig} component provided by
that package will be loaded and incorporated into the schema being
used to load the configuration file.  After the import, section types
defined in the component may be used in the configuration.

More detail is needed for this to really make sense.

A schema may define section types which are \emph{abstract}; these
cannot be used directly in a configuration, but multiple concrete
section types can be defined which \emph{implement} the abstract
types.  Wherever the application allows an abstract type to be used,
any concrete type which implements that abstract type can be used in
an actual configuration.

The \keyword{\%import} directive allows loading schema components
which provide alternate concrete section types which implement the
abstract types defined by the application.  This allows third-party
implementations of abstract types to be used in place of or in
addition to implementations provided with the application.

Consider an example application application which supports logging in
the same way Zope 2 does.  There are some parameters which configure
the general behavior of the logging mechanism, and an arbitrary number
of \emph{log handlers} may be specified to control how the log
messages are handled.  Several log handlers are provided by the
application.  Here is an example logging configuration:

\begin{verbatim}
<eventlog>
  level verbose

  <logfile>
    path /var/log/myapp/events.log
  </logfile>
</eventlog>
\end{verbatim}

A third-party component may provide a log handler to send
high-priority alerts the system administrator's text pager or
SMS-capable phone.  All that's needed is to install the implementation
so it can be imported by Python, and modify the configuration:

\begin{verbatim}
%import my.pager.loghandler

<eventlog>
  level verbose

  <logfile>
    path /var/log/myapp/events.log
  </logfile>

  <pager>
    number   1-800-555-1234
    message  Something broke!
  </pager>
</eventlog>
\end{verbatim}


\subsection{Textual Substitution in Values}

\module{ZConfig} provides a limited way to re-use portions of a value
using simple string substitution.  To use this facility, define named
bits of replacement text using the \keyword{\%define} directive, and
reference these texts from values.

The syntax for \keyword{\%define} is:

\begin{alltt}
%define \var{name} \optional{\var{value}}
\end{alltt}

The value of \var{name} must be a sequence of letters, digits, and
underscores, and may not start with a digit; the namespace for these
names is separate from the other namespaces used with
\module{ZConfig}, and is case-insensitive.  If \var{value} is
omitted, it will be the empty string.  If given, there must be
whitespace between \var{name} and \var{value}; \var{value} will not
include any whitespace on either side, just like values from key-value
pairs.

Names must be defined before they are used, and may not be
re-defined.  All resources being parsed as part of a configuration
share a single namespace for defined names.  This means that resources
which may be included more than once should not define any names.

References to defined names from configuration values use the syntax
described for the \refmodule{ZConfig.substitution} module.
Configuration values which include a \character{\$} as part of the
actual value will need to use \code{\$\$} to get a single
\character{\$} in the result.

The values of defined names are processed in the same way as
configuration values, and may contain references to named
definitions.

For example, the value for \code{key} will evaluate to \code{value}:

\begin{verbatim}
%define name value
key $name
\end{verbatim} %$ <-- bow to font-lock


\section{Writing Configuration Schema \label{writing-schema}}

\module{ZConfig} schema are written as XML documents.

Data types are searched in a special namespace defined by the data
type registry.  The default registry has slightly magical semantics:
If the value can be matched to a standard data type when interpreted
as a \datatype{basic-key}, the standard data type will be used.  If
that fails, the value must be a \datatype{dotted-name} containing at
least one dot, and a conversion function will be sought using the
\method{search()} method of the data type registry used to load the
schema.


\subsection{Schema Elements \label{elements}}

For each element, the content model is shown, followed by a
description of how the element is used, and then a list of the
available attributes.  For each attribute, the type of the value is
given as either the name of a \module{ZConfig} datatype or an XML
attribute value type.  Familiarity with XML's Document Type Definition
language is helpful.

The following elements are used to describe a schema:

\begin{elementdesc}{schema}{description?, metadefault?, example?,
                            import*,
                            (sectiontype | abstracttype)*,
                            (section | key | multisection |
                            multikey)*}
  Document element for a \module{ZConfig} schema.

  \begin{attributedesc}{extends}{\datatype{space-separated-url-references}}
  A list of URLs of base schemas from which this section type will inherit key,
  section, and section type declarations.  If omitted, this schema
  is defined using only the keys, sections, and section types contained within
  the \element{schema} element.
  \end{attributedesc}

  \begin{attributedesc}{datatype}{\datatype{basic-key}
                                  or \datatype{dotted-name}}
    The data type converter which will be applied to the value of this
    section.  If the value is a \datatype{dotted-name} that begins
    with a period, the value of \attribute{prefix} will be pre-pended,
    if set.  If any base schemas are listed in the \attribute{extends}
    attribute, the default value for this attribute comes from the base
    schemas.  If the base schemas all use the same \attribute{datatype}, then
    that data type will be the default value for the extending schema.  If
    there are no base schemas, the default value is \datatype{null}, which
    means that the \module{ZConfig} section object will be used unconverted.
    If the base schemas have different \attribute{datatype} definitions, you
    must explicitly define the \attribute{datatype} in the extending schema.
  \end{attributedesc}

  \begin{attributedesc}{handler}{\datatype{basic-key}}
  \end{attributedesc}

  \begin{attributedesc}{keytype}{\datatype{basic-key}
                                  or \datatype{dotted-name}}
    The data type converter which will be applied to keys found in
    this section.  This can be used to constrain key values in
    different ways; two data types which may be especially useful are
    the \datatype{identifier} and \datatype{ipaddr-or-hostname}
    types.  If the value is a \datatype{dotted-name} that begins
    with a period, the value of \attribute{prefix} will be pre-pended,
    if set.  If any base schemas are listed in the \attribute{extends}
    attribute, the default value for this attribute comes from the base
    schemas.  If the base schemas all use the same \attribute{keytype}, then
    that key type will be the default value for the extending schema.  If there
    are no base schemas, the default value is \datatype{basic-key}.  If the
    base schemas have different \attribute{keytype} definitions, you must
    explicitly define the \attribute{keytype} in the extending schema.
  \end{attributedesc}

  \begin{attributedesc}{prefix}{\datatype{dotted-name}}
    Prefix to be pre-pended in front of partial dotted-names that
    start with a period.  The value of this attribute is used in all
    contexts with the \element{schema} element if it hasn't been
    overridden by an inner element with a \attribute{prefix}
    attribute.
  \end{attributedesc}
\end{elementdesc}

\begin{elementdesc}{description}{PCDATA}
  Descriptive text explaining the purpose the container of the
  \element{description} element.  Most other elements can contain
  a \element{description} element as their first child.
  At most one \element{description} element may appear in a given
  context.

  \begin{attributedesc}{format}{NMTOKEN}
    Optional attribute that can be added to indicate what conventions
    are used to mark up the contained text.  This is intended to serve
    as a hint for documentation extraction tools.  Suggested values
    are:

    \begin{tableii}{l|l}{code}{Value}{Content Format}
      \lineii{plain}{\mimetype{text/plain}; blank lines separate paragraphs}
      \lineii{rest}{reStructuredText}
      \lineii{stx}{Classic Structured Text}
    \end{tableii}
  \end{attributedesc}
\end{elementdesc}

\begin{elementdesc}{example}{PCDATA}
  An example value.  This serves only as documentation.
\end{elementdesc}

\begin{elementdesc}{metadefault}{PCDATA}
  A description of the default value, for human readers.  This may
  include information about how a computed value is determined when
  the schema does not specify a default value.
\end{elementdesc}

\begin{elementdesc}{abstracttype}{description?}
  Define an abstract section type.

  \begin{attributedesc}{name}{\datatype{basic-key}}
    The name of the abstract section type; required.
  \end{attributedesc}
\end{elementdesc}

\begin{elementdesc}{sectiontype}{description?, (section | key |
                                  multisection | multikey)*}
  Define a concrete section type.

  \begin{attributedesc}{datatype}{\datatype{basic-key}
                                  or \datatype{dotted-name}}
    The data type converter which will be applied to the value of this
    section.  If the value is a \datatype{dotted-name} that begins
    with a period, the value of \attribute{prefix} will be pre-pended,
    if set.  If \attribute{datatype} is omitted and
    \attribute{extends} is used, the \attribute{datatype} from the
    section type identified by the \attribute{extends} attribute is
    used.
  \end{attributedesc}

  \begin{attributedesc}{extends}{\datatype{basic-key}}
    The name of a concrete section type from which this section type
    acquires all key and section declarations.  This type does
    \emph{not} automatically implement any abstract section type
    implemented by the named section type.  If omitted, this section
    is defined with only the keys and sections contained within the
    \element{sectiontype} element.  The new section type is called a
    \emph{derived} section type, and the type named by this attribute
    is called the \emph{base} type.  Values for the
    \attribute{datatype} and \attribute{keytype} attributes are
    acquired from the base type if not specified.
  \end{attributedesc}

  \begin{attributedesc}{implements}{\datatype{basic-key}}
    The name of an abstract section type which this concrete section
    type implements.  If omitted, this section type does not implement
    any abstract type, and can only be used if it is specified
    directly in a schema or other section type.
  \end{attributedesc}

  \begin{attributedesc}{keytype}{\datatype{basic-key}}
    The data type converter which will be applied to keys found in
    this section.  This can be used to constrain key values in
    different ways; two data types which may be especially useful are
    the \datatype{identifier} and \datatype{ipaddr-or-hostname}
    types.  If the value is a \datatype{dotted-name} that begins
    with a period, the value of \attribute{prefix} will be pre-pended,
    if set.  The default value is \datatype{basic-key}.  If
    \attribute{keytype} is omitted and \attribute{extends} is used,
    the \attribute{keytype} from the section type identified by the
    \attribute{extends} attribute is used.
  \end{attributedesc}

  \begin{attributedesc}{name}{\datatype{basic-key}}
    The name of the section type; required.
  \end{attributedesc}

  \begin{attributedesc}{prefix}{\datatype{dotted-name}}
    Prefix to be pre-pended in front of partial dotted-names that
    start with a period.  The value of this attribute is used in all
    contexts in the \element{sectiontype} element.  If omitted, the
    prefix specified by a containing context is used if specified.
  \end{attributedesc}
\end{elementdesc}

\begin{elementdesc}{import}{EMPTY}
  Import a schema component.  Exactly one of the attributes
  \attribute{package} and \attribute{src} must be specified.

  \begin{attributedesc}{file}{file name without directory information}
    Name of the component file within a package; if not specified,
    \file{component.xml} is used.  This may only be given when
    \attribute{package} is used.  (The \file{component.xml} file is
    always used when importing via \keyword{\%import} from a
    configuration file.)
  \end{attributedesc}

  \begin{attributedesc}{package}{\datatype{dotted-suffix}}
    Name of a Python package that contains the schema component being
    imported.  The component will be loaded from the file identified
    by the \attribute{file} attribute, or \file{component.xml} if
    \attribute{file} is not specified.  If the package name given
    starts with a dot (\character{.}), the name used will be the
    current prefix and the value of this attribute concatenated.
  \end{attributedesc}

  \begin{attributedesc}{src}{\datatype{url-reference}}
    URL to a separate schema which can provide useful types.  The
    referenced resource must contain a schema, not a schema
    component.  Section types defined or imported by the referenced
    schema are added to the schema containing the \element{import};
    top-level keys and sections are ignored.
  \end{attributedesc}
\end{elementdesc}

\begin{elementdesc}{key}{description?, example?, metadefault?, default*}
  A \element{key} element is used to describe a key-value pair which
  may occur at most once in the section type or top-level schema in
  which it is listed.

  \begin{attributedesc}{attribute}{\datatype{identifier}}
    The name of the Python attribute which this key should be the
    value of on a \class{SectionValue} instance.  This must be unique
    within the immediate contents of a section type or schema.  If
    this attribute is not specified, an attribute name will be
    computed by converting hyphens in the key name to underscores.
  \end{attributedesc}

  \begin{attributedesc}{datatype}{\datatype{basic-key}
                                  or \datatype{dotted-name}}
    The data type converter which will be applied to the value of this
    key.  If the value is a \datatype{dotted-name} that begins
    with a period, the value of \attribute{prefix} will be pre-pended,
    if set.
  \end{attributedesc}

  \begin{attributedesc}{default}{\datatype{string}}
    If the key-value pair is optional and this attribute is specified,
    the value of this attribute will be converted using the appropriate
    data type converter and returned to the application as the
    configured value.  This attribute may not be specified if the
    \attribute{required} attribute is \code{yes}.
  \end{attributedesc}

  \begin{attributedesc}{handler}{\datatype{basic-key}}
  \end{attributedesc}

  \begin{attributedesc}{name}{\datatype{basic-key}}
    The name of the key, as it must be given in a configuration
    instance, or `\code{*}'.  If the value is `\code{*}', any name not
    already specified as a key may be used, and the configuration
    value for the key will be a dictionary mapping from the key name
    to the value.  In this case, the \attribute{attribute} attribute
    must be specified, and the data type for the key will be applied
    to each key which is found.
  \end{attributedesc}

  \begin{attributedesc}{required}{\code{yes|no}}
    Specifies whether the configuration instance is required to
    provide the key.  If the value is \code{yes}, the
    \attribute{default} attribute may not be specified and an error
    will be reported if the configuration instance does not specify a
    value for the key.  If the value is \code{no} (the default) and
    the configuration instance does not specify a value, the value
    reported to the application will be that specified by the
    \attribute{default} attribute, if given, or \code{None}.
  \end{attributedesc}
\end{elementdesc}


\begin{elementdesc}{multikey}{description?, example?, metadefault?, default*}
  A \element{multikey} element is used to describe a key-value pair
  which may occur any number of times in the section type or top-level
  schema in which it is listed.

  \begin{attributedesc}{attribute}{\datatype{identifier}}
    The name of the Python attribute which this key should be the
    value of on a \class{SectionValue} instance.  This must be unique
    within the immediate contents of a section type or schema.  If
    this attribute is not specified, an attribute name will be
    computed by converting hyphens in the key name to underscores.
  \end{attributedesc}

  \begin{attributedesc}{datatype}{\datatype{basic-key}
                                  or \datatype{dotted-name}}
    The data type converter which will be applied to the value of this
    key.  If the value is a \datatype{dotted-name} that begins
    with a period, the value of \attribute{prefix} will be pre-pended,
    if set.
  \end{attributedesc}

  \begin{attributedesc}{handler}{\datatype{basic-key}}
  \end{attributedesc}

  \begin{attributedesc}{name}{\datatype{basic-key}}
    The name of the key, as it must be given in a configuration
    instance, or `\code{+}'.  If the value is `\code{+}', any name not
    already specified as a key may be used, and the configuration
    value for the key will be a dictionary mapping from the key name
    to the value.  In this case, the \attribute{attribute} attribute
    must be specified, and the data type for the key will be applied
    to each key which is found.
  \end{attributedesc}

  \begin{attributedesc}{required}{\code{yes|no}}
    Specifies whether the configuration instance is required to
    provide the key.  If the value is \code{yes}, no \element{default}
    elements may be specified and an error will be reported if the
    configuration instance does not specify at least one value for the
    key.  If the value is \code{no} (the default) and the
    configuration instance does not specify a value, the value
    reported to the application will be a list containing one element
    for each \element{default} element specified as a child of the
    \element{multikey}.  Each value will be individually converted
    according to the \attribute{datatype} attribute.
  \end{attributedesc}
\end{elementdesc}


\begin{elementdesc}{default}{PCDATA}
  Each \element{default} element specifies a single default value for
  a \element{multikey}.  This element can be repeated to produce a
  list of individual default values.  The text contained in the
  element will be passed to the datatype conversion for the
  \element{multikey}.

  \begin{attributedesc}{key}{key type of the containing sectiontype}
    Key to associate with the default value.  This is only used for
    defaults of a \element{key} or \element{multikey} with a
    \attribute{name} of \code{+}; in that case this attribute is
    required.  It is an error to use the \attribute{key} attribute
    with a \element{default} element for a \element{multikey} with a
    name other than \code{+}.

    \begin{notice}[warning]
      The datatype of this attribute is that of the section type
      \emph{containing} the actual keys, not necessarily that of the
      section type which defines the key.  If a derived section
      overrides the key type of the base section type, the actual
      key type used is that of the derived section.

      This can lead to confusing errors in schemas, though the
      \refmodule{ZConfig} package checks for this when the schema is
      loaded.  This situation is particularly likely when a derived
      section type uses a key type which collapses multiple default
      keys which were not collapsed by the base section type.

      Consider this example schema:

\begin{verbatim}
<schema>
  <sectiontype name="base" keytype="identifier">
    <key name="+" attribute="mapping">
      <default key="foo">some value</default>
      <default key="FOO">some value</default>
    </key>
  </sectiontype>

  <sectiontype name="derived" keytype="basic-key"
               extends="base"/>

  <section type="derived" name="*" attribute="section"/>
</schema>
\end{verbatim}

      When this schema is loaded, a set of defaults for the
      \datatype{derived} section type is computed.  Since
      \datatype{basic-key} is case-insensitive (everything is
      converted to lower case), \samp{foo} and \samp{Foo} are both
      converted to \samp{foo}, which clashes since \element{key} only
      allows one value for each key.
    \end{notice}
  \end{attributedesc}
\end{elementdesc}


\begin{elementdesc}{section}{description?}
  A \element{section} element is used to describe a section which may
  occur at most once in the section type or top-level schema in which
  it is listed.

  \begin{attributedesc}{attribute}{\datatype{identifier}}
    The name of the Python attribute which this section should be the
    value of on a \class{SectionValue} instance.  This must be unique
    within the immediate contents of a section type or schema.  If
    this attribute is not specified, an attribute name will be
    computed by converting hyphens in the section name to underscores,
    in which case the \attribute{name} attribute may not be \code{*}
    or \code{+}.
  \end{attributedesc}

  \begin{attributedesc}{handler}{\datatype{basic-key}}
  \end{attributedesc}

  \begin{attributedesc}{name}{\datatype{basic-key}}
    The name of the section, as it must be given in a configuration
    instance, \code{*}, or \code{+}.  If the value is \code{*}, any
    name not already specified as a key may be used.  If the value is
    \code{*} or \code{+}, the \attribute{attribute} attribute must be
    specified.  If the value is \code{*}, any name is allowed, or the
    name may be omitted.  If the value is \code{+}, any name is
    allowed, but some name must be provided.
  \end{attributedesc}

  \begin{attributedesc}{required}{\code{yes|no}}
    Specifies whether the configuration instance is required to
    provide the section.  If the value is \code{yes}, an error will be
    reported if the configuration instance does not include the
    section.  If the value is \code{no} (the default) and the
    configuration instance does not include the section, the value
    reported to the application will be \code{None}.
  \end{attributedesc}

  \begin{attributedesc}{type}{\datatype{basic-key}}
    The section type which matching sections must implement.  If the
    value names an abstract section type, matching sections in the
    configuration file must be of a type which specifies that it
    implements the named abstract type.  If the name identifies a
    concrete type, the section type must match exactly.
  \end{attributedesc}
\end{elementdesc}


\begin{elementdesc}{multisection}{description?}
  A \element{multisection} element is used to describe a section which
  may occur any number of times in the section type or top-level
  schema in which it is listed.

  \begin{attributedesc}{attribute}{\datatype{identifier}}
    The name of the Python attribute which matching sections should be
    the value of on a \class{SectionValue} instance.  This is required
    and must be unique within the immediate contents of a section type
    or schema.  The \class{SectionValue} instance will contain a list
    of matching sections.
  \end{attributedesc}

  \begin{attributedesc}{handler}{\datatype{basic-key}}
  \end{attributedesc}

  \begin{attributedesc}{name}{\datatype{basic-key}}
    For a \element{multisection}, any name not already specified as a
    key may be used.  If the value is \code{*} or \code{+}, the
    \attribute{attribute} attribute must be specified.  If the value
    is \code{*}, any name is allowed, or the name may be omitted.  If
    the value is \code{+}, any name is allowed, but some name must be
    provided.  No other value for the \attribute{name} attribute is
    allowed for a \element{multisection}.
  \end{attributedesc}

  \begin{attributedesc}{required}{\code{yes|no}}
    Specifies whether the configuration instance is required to
    provide at least one matching section.  If the value is
    \code{yes}, an error will be reported if the configuration
    instance does not include the section.  If the value is \code{no}
    (the default) and the configuration instance does not include the
    section, the value reported to the application will be
    \code{None}.
  \end{attributedesc}

  \begin{attributedesc}{type}{\datatype{basic-key}}
    The section type which matching sections must implement.  If the
    value names an abstract section type, matching sections in the
    configuration file must be of types which specify that they
    implement the named abstract type.  If the name identifies a
    concrete type, the section type must match exactly.
  \end{attributedesc}
\end{elementdesc}


\subsection{Schema Components \label{schema-components}}

XXX need more explanation

\module{ZConfig} supports schema components that can be
provided by disparate components, and allows them to be knit together
into concrete schema for applications.  Components cannot add
additional keys or sections in the application schema.

A schema \dfn{component} is allowed to define new abstract and
section types.
Components are identified using a dotted-name, similar to a Python
module name.  For example, one component may be \code{zodb.storage}.

Schema components are stored alongside application code since they
directly reference datatype code.  Schema components are provided by
Python packages.  The component definition is normally stored in the
file \file{component.xml}; an alternate filename may be specified
using the \attribute{file} attribute of the \element{import} element.
Components imported using the \keyword{\%import} keyword from a
configuration file must be named \file{component.xml}.
The component defines the types provided by that component; it must
have a \element{component} element as the document element.

The following element is used as the document element for schema
components.  Note that schema components do not allow keys and
sections to be added to the top-level of a schema; they serve only to
provide type definitions.

\begin{elementdesc}{component}{description?, (abstracttype | sectiontype)*}
  The top-level element for schema components.

  \begin{attributedesc}{prefix}{\datatype{dotted-name}}
    Prefix to be pre-pended in front of partial dotted-names that
    start with a period.  The value of this attribute is used in all
    contexts within the \element{component} element if it hasn't been
    overridden by an inner element with a \attribute{prefix}
    attribute.
  \end{attributedesc}
\end{elementdesc}


\section{Standard \module{ZConfig} Datatypes\label{standard-datatypes}}

There are a number of data types which can be identified using the
\attribute{datatype} attribute on \element{key},
\element{sectiontype}, and \element{schema} elements.
Applications may extend the set of datatypes by calling the
\method{register()} method of the data type registry being used or by
using Python dotted-names to refer to conversion routines defined in
code.

The following data types are provided by the default type registry.

\begin{definitions}
\term{\datatype{basic-key}}
  The default data type for a key in a ZConfig configuration file.
  The result of conversion is always lower-case, and matches the
  regular expression \regexp{[a-z][-._a-z0-9]*}.

\term{\datatype{boolean}}
  Convert a human-friendly string to a boolean value.  The names
  \code{yes}, \code{on}, and \code{true} convert to \constant{True},
  while \code{no}, \code{off}, and \code{false} convert to
  \constant{False}.  Comparisons are case-insensitive.  All other
  input strings are disallowed.

\term{\datatype{byte-size}}
  A specification of a size, with byte multiplier suffixes (for
  example, \samp{128MB}).  Suffixes are case insensitive and may be
  \samp{KB}, \samp{MB}, or \samp{GB}

\term{\datatype{dotted-name}}
  A string consisting of one or more \datatype{identifier} values
  separated by periods (\character{.}).

\term{\datatype{dotted-suffix}}
  A string consisting of one or more \datatype{identifier} values
  separated by periods (\character{.}), possibly prefixed by a
  period.  This can be used to indicate a dotted name that may be
  specified relative to some base dotted name.

\term{\datatype{existing-dirpath}}
  Validates that the directory portion of a pathname exists.  For
  example, if the value provided is \file{/foo/bar}, \file{/foo} must
  be an existing directory.  No conversion is performed.

\term{\datatype{existing-directory}}
  Validates that a directory by the given name exists on 
  the local filesystem.  No conversion is performed. 

\term{\datatype{existing-file}}
  Validates that a file by the given name exists.  No conversion 
  is performed. 

\term{\datatype{existing-path}}
  Validates that a path (file, directory, or symlink) by the
  given name exists on the local filesystem.  No conversion
  is performed.

\term{\datatype{float}}
  A Python float.  \code{Inf}, \code{-Inf}, and \code{NaN} are not
  allowed.

\term{\datatype{identifier}}
  Any valid Python identifier.

\term{\datatype{inet-address}}
  An Internet address expressed as a \code{(\var{hostname},
  \var{port})} pair.  If only the port is specified, the default host
  will be returned for \var{hostname}.  The default host is
  \code{localhost} on Windows and the empty string on all other
  platforms.  If the port is omitted, \code{None} will be returned for
  \var{port}.

\term{\datatype{integer}}
  Convert a value to an integer.  This will be a Python \class{int} if
  the value is in the range allowed by \class{int}, otherwise a Python
  \class{long} is returned.

\term{\datatype{ipaddr-or-hostname}}
  Validates a valid IP address or hostname.  If the first 
  character is a digit, the value is assumed to be an IP 
  address.  If the first character is not a digit, the value 
  is assumed to be a hostname.  Hostnames are converted to lower
  case.

\term{\datatype{locale}}
  Any valid locale specifier accepted by the available
  \function{locale.setlocale()} function.  Be aware that only the
  \code{'C'} locale is supported on some platforms.

\term{\datatype{null}}
  No conversion is performed; the value passed in is the value
  returned.  This is the default data type for section values.

\term{\datatype{port-number}}
  Returns a valid port number as an integer.  Validity does not imply
  that any particular use may be made of the port, however.  For
  example, port number lower than 1024 generally cannot be bound by
  non-root users.

\term{\datatype{socket-address}}
  An address for a socket.  The converted value is an object providing
  two attributes.  \member{family} specifies the address family
  (\constant{AF_INET} or \constant{AF_UNIX}), with \code{None} instead
  of \constant{AF_UNIX} on platforms that don't support it.  The
  \member{address} attribute will be the address that should be passed
  to the socket's \method{bind()} method.  If the family is
  \constant{AF_UNIX}, the specific address will be a pathname; if the
  family is \constant{AF_INET}, the second part will be the result of
  the \datatype{inet-address} conversion.

\term{\datatype{string}}
  Returns the input value as a string.  If the source is a Unicode
  string, this implies that it will be checked to be simple 7-bit
  \ASCII.  This is the default data type for values in
  configuration files.

\term{\datatype{time-interval}}
  A specification of a time interval in seconds, with multiplier
  suffixes (for example, \code{12h}).  Suffixes are case insensitive
  and may be \samp{s} (seconds), \samp{m} (minutes), \samp{h} (hours),
  or \samp{d} (days).

\term{\datatype{timedelta}}
  Similar to the \datatype{time-interval}, this data type returns a Python
  datetime.timedelta object instead of a float.  The set of suffixes
  recognized by \datatype{timedelta} are: \samp{w} (weeks), \samp{d} (days),
  \samp{h} (hours), \samp{m} (minutes), \samp{s} (seconds).  Values may be
  floats, for example: \code{4w 2.5d 7h 12m 0.001s}.

\end{definitions}


\section{Standard \module{ZConfig} Schema Components
         \label{standard-components}}

\module{ZConfig} provides a few convenient schema components as part
of the package.  These may be used directly or can server as examples
for creating new components.


\subsection{\module{ZConfig.components.basic}}

The \module{ZConfig.components.basic} package provides small
components that can be helpful in composing application-specific
components and schema.  There is no large functionality represented by
this package.  The default component provided by this package simply
imports all of the smaller components.  This can be imported using

\begin{verbatim}
<import package="ZConfig.components.basic"/>
\end{verbatim}

Each of the smaller components is documented directly; importing these
selectively can reduce the time it takes to load a schema slightly,
and allows replacing the other basic components with alternate
components (by using different imports that define the same type
names) if desired.


\subsubsection{The Mapping Section Type \label{basic-mapping}}

There is a basic section type that behaves like a simple Python
mapping; this can be imported directly using

\begin{verbatim}
<import package="ZConfig.components.basic" file="mapping.xml"/>
\end{verbatim}

This defines a single section type, \datatype{ZConfig.basic.mapping}.
When this is used, the section value is a Python dictionary mapping
keys to string values.

This type is intended to be used by extending it in simple ways.  The
simplest is to create a new section type name that makes more sense
for the application:

\begin{verbatim}
<import package="ZConfig.components.basic" file="mapping.xml"/>

<sectiontype name="my-mapping"
             extends="ZConfig.basic.mapping"
             />

<section name="*"
         type="my-mapping"
         attribute="map"
         />
\end{verbatim}

This allows a configuration to contain a mapping from
\datatype{basic-key} names to string values like this:

\begin{verbatim}
<my-mapping>
  This that
  and the other
</my-mapping>
\end{verbatim}

The value of the configuration object's \member{map} attribute would
then be the dictionary

\begin{verbatim}
{'this': 'that',
 'and': 'the other',
 }
\end{verbatim}

(Recall that the \datatype{basic-key} data type converts everything to
lower case.)

Perhaps a more interesting application of
\datatype{ZConfig.basic.mapping} is using the derived type to override
the \attribute{keytype}.  If we have the conversion function:

\begin{verbatim}
def email_address(value):
    userid, hostname = value.split("@", 1)
    hostname = hostname.lower()  # normalize what we know we can
    return "%s@%s" % (userid, hostname)
\end{verbatim}

then we can use this as the key type for a derived mapping type:

\begin{verbatim}
<import package="ZConfig.components.basic" file="mapping.xml"/>

<sectiontype name="email-users"
             extends="ZConfig.basic.mapping"
             keytype="mypkg.datatypes.email_address"
             />

<section name="*"
         type="email-users"
         attribute="email_users"
         />
\end{verbatim}


\subsection{\module{ZConfig.components.logger}}

The \module{ZConfig.components.logger} package provides configuration
support for the \ulink{\module{logging} package}
{http://docs.python.org/lib/module-logging.html} in
Python's standard library.  This component can be imported using

\begin{verbatim}
<import package="ZConfig.components.logger"/>
\end{verbatim}

This component defines two abstract types and several concrete section
types.  These can be imported as a unit, as above, or as four smaller
components usable in creating alternate logging packages.

The first of the four smaller components contains the abstract types,
and can be imported using

\begin{verbatim}
<import package="ZConfig.components.logger" file="abstract.xml"/>
\end{verbatim}

The two abstract types imported by this are:

\begin{definitions}

\term{\datatype{ZConfig.logger.log}}
  Logger objects are represented by this abstract type.

\term{\datatype{ZConfig.logger.handler}}
  Each logger object can have one or more ``handlers'' associated with
  them.  These handlers are responsible for writing logging events to
  some form of output stream using appropriate formatting.  The output
  stream may be a file on a disk, a socket communicating with a server
  on another system, or a series of \code{syslog} messages.  Section
  types which implement this type represent these handlers.

\end{definitions}


The second and third of the smaller components provides section types
that act as factories for \class{logging.Logger} objects.  These can be
imported using

\begin{verbatim}
<import package="ZConfig.components.logger" file="eventlog.xml"/>
<import package="ZConfig.components.logger" file="logger.xml"/>
\end{verbatim}

The types defined in these components implement the
\datatype{ZConfig.logger.log} abstract type.  The \file{eventlog.xml}
component defines an \datatype{eventlog} type which represents the
root logger from the the \module{logging} package (the return value of
\function{logging.getLogger()}), while the \file{logger.xml} component
defines a \datatype{logger} section type which represents a named
logger (as returned by \function{logging.getLogger(\var{name})}).


The third of the smaller components provides section types that are
factories for \class{logging.Handler} objects.  This can be imported
using

\begin{verbatim}
<import package="ZConfig.components.logger" file="handlers.xml"/>
\end{verbatim}

The types defined in this component implement the
\datatype{ZConfig.logger.handler} abstract type.



The configuration objects provided by both the logger and handler
types are factories for the finished loggers and handlers.  These
factories should be called with no arguments to retrieve the logger or
log handler objects.  Calling the factories repeatedly will cause the
same objects to be returned each time, so it's safe to simply call
them to retrieve the objects.

The factories for the logger objects, whether the \datatype{eventlog}
or \datatype{logger} section type is used, provide a \method{reopen()}
method which may be called to close any log files and re-open them.
This is useful when using a \UNIX{} signal to effect log file
rotation: the signal handler can call this method, and not have to
worry about what handlers have been registered for the logger.

Building an application that uses the logging components is fairly
straightforward.  The schema needs to import the relevant components
and declare their use:

\begin{verbatim}
<schema>
  <import package="ZConfig.components.logger" file="eventlog.xml"/>
  <import package="ZConfig.components.logger" file="handlers.xml"/>

  <section type="eventlog" name="*" attribute="eventlog"
           required="yes"/>
</schema>
\end{verbatim}

In the application, the schema and configuration file should be loaded
normally.  Once the configuration object is available, the logger
factory should be called to configure Python's \module{logging} package:

\begin{verbatim}
import os
import ZConfig

def run(configfile):
    schemafile = os.path.join(os.path.dirname(__file__), "schema.xml")
    schema = ZConfig.loadSchema(schemafile)
    config, handlers = ZConfig.loadConfig(schema, configfile)

    # configure the logging package:
    config.eventlog()

    # now do interesting things
\end{verbatim}

An example configuration file for this application may look like this:

\begin{verbatim}
<eventlog>
  level  info

  <logfile>
    path        /var/log/myapp
    format      %(asctime)s %(levelname)s %(name)s %(message)s
    # locale-specific date/time representation
    dateformat  %c
  </logfile>

  <syslog>
    level    error
    address  syslog.example.net:514
    format   %(levelname)s %(name)s %(message)s
  </syslog>
</eventlog>
\end{verbatim}

Refer to the \module{logging} package documentation for the names
available in the message format strings (the \code{format} key in the
log handlers).  The date format strings (the \code{dateformat} key in
the log handlers) are the same as those accepted by the
\function{time.strftime()} function.


\begin{seealso}
  \seepep{282}{A Logging System}
         {The proposal which described the logging feature for
          inclusion in the Python standard library.}
  \seelink{http://docs.python.org/lib/module-logging.html}
          {\module{logging} --- Logging facility for Python}
          {Python's \module{logging} package documentation, from the
           \citetitle[http://docs.python.org/lib/lib.html]
           {Python Library Reference}.}
  \seelink{http://www.red-dove.com/python_logging.html}
          {Original Python \module{logging} package}
          {This is the original source for the \module{logging}
           package.  This is mostly of historical interest.}
\end{seealso}


\section{Using Components to Extend Schema}

% XXX This section needs a lot of work, but should get people started
% who really want to add new pieces to ZConfig-configured applications.

It is possible to use schema components and the \keyword{\%import}
construct to extend the set of section types available for a specific
configuration file, and allow the new components to be used in place
of standard components.

The key to making this work is the use of abstract section types.
Wherever the original schema accepts an abstract type, it is possible
to load new implementations of the abstract type and use those instead
of, or in addition to, the implementations loaded by the original
schema.

Abstract types are generally used to represent interfaces.  Sometimes
these are interfaces for factory objects, and sometimes not, but
there's an interface that the new component needs to implement.  What
interface is required should be documented in the
\element{description} element in the \element{abstracttype} element;
this may be by reference to an interface specified in a Python module
or described in some other bit of documentation.

The following things need to be created to make the new component
usable from the configuration file:

\begin{enumerate}
  \item An implementation of the required interface.

  \item A schema component that defines a section type that contains
        the information needed to construct the component.

  \item A ``datatype'' function that converts configuration data to an
        instance of the component.
\end{enumerate}

For simplicity, let's assume that the implementation is defined by a
Python class.

The example component we build here will be in the \module{noise}
package, but any package will do.  Components loadable using
\keyword{\%import} must be contained in the \file{component.xml} file;
alternate filenames may not be selected by the \keyword{\%import}
construct.

Create a ZConfig component that provides a section type to support
your component.  The new section type must declare that it implements
the appropriate abstract type; it should probably look something like
this:

\begin{verbatim}
<component prefix="noise.server">
  <import package="ZServer"/>

  <sectiontype name="noise-generator"
               implements="ZServer.server"
               datatype=".NoiseServerFactory">

    <!-- specific configuration data should be described here -->

    <key name="port"
         datatype="port-number"
         required="yes">
      <description>
        Port number to listen on.
      </description>
    </key>

    <key name="color"
         datatype=".noise_color"
         default="white">
      <description>
        Silly way to specify a noise generation algorithm.
      </description>
    </key>

  </sectiontype>
</component>
\end{verbatim}

This example uses one of the standard ZConfig datatypes,
\datatype{port-number}, and requires two additional types to be
provided by the \module{noise.server} module:
\class{NoiseServerFactory} and \function{noise_color()}.

The \function{noise_color()} function is a datatype conversion for a
key, so it accepts a string and returns the value that should be used:

\begin{verbatim}
_noise_colors = {
    # color -> r,g,b
    'white': (255, 255, 255),
    'pink':  (255, 182, 193),
    }

def noise_color(string):
    if string in _noise_colors:
        return _noise_colors[string]
    else:
        raise ValueError('unknown noise color: %r' % string)
\end{verbatim}

\class{NoiseServerFactory} is a little different, as it's the datatype
function for a section rather than a key.  The parameter isn't a
string, but a section value object with two attributes, \member{port}
and \member{color}.

Since the \datatype{ZServer.server} abstract type requires that the
component returned is a factory object, the datatype function can be
implemented at the constructor for the class of the factory object.
(If the datatype function could select different implementation
classes based on the configuration values, it makes more sense to use
a simple function that returns the appropriate implementation.)

A class that implements this datatype might look like this:

\begin{verbatim}
from ZServer.datatypes import ServerFactory
from noise.generator import WhiteNoiseGenerator, PinkNoiseGenerator

class NoiseServerFactory(ServerFactory):

    def __init__(self, section):
        # host and ip will be initialized by ServerFactory.prepare()
        self.host = None
        self.ip = None
        self.port = section.port
        self.color = section.color

    def create(self):
        if self.color == 'white':
            generator = WhiteNoiseGenerator()
        else:
            generator = PinkNoiseGenerator()
        return NoiseServer(self.ip, self.port, generator)
\end{verbatim}

You'll need to arrange for the package containing this component is
available on Python's \code{sys.path} before the configuration file is
loaded; this is mostly easily done by manipulating the
\envvar{PYTHONPATH} environment variable.

Your configuration file can now include the following to load and use
your new component:

\begin{verbatim}
%import noise

<noise-generator>
  port 1234
  color white
</noise-generator>
\end{verbatim}


\section{\module{ZConfig} --- Basic configuration support}

\declaremodule{}{ZConfig}
\modulesynopsis{Configuration package.}

The main \module{ZConfig} package exports these convenience functions:

\begin{funcdesc}{loadConfig}{schema, url\optional{, overrides}}
  Load and return a configuration from a URL or pathname given by
  \var{url}.  \var{url} may be a URL, absolute pathname, or relative
  pathname.  Fragment identifiers are not supported.  \var{schema} is
  a reference to a schema loaded by \function{loadSchema()} or
  \function{loadSchemaFile()}.
  The return value is a tuple containing the configuration object and
  a composite handler that, when called with a name-to-handler
  mapping, calls all the handlers for the configuration.

  The optional \var{overrides} argument represents information derived
  from command-line arguments.  If given, it must be either a sequence
  of value specifiers, or \code{None}.  A \dfn{value specifier} is a
  string of the form \code{\var{optionpath}=\var{value}}.  The
  \var{optionpath} specifies the ``full path'' to the configuration
  setting: it can contain a sequence of names, separated by
  \character{/} characters. Each name before the last names a section
  from the configuration file, and the last name corresponds to a key
  within the section identified by the leading section names.  If
  \var{optionpath} contains only one name, it identifies a key in the
  top-level schema.  \var{value} is a string that will be treated
  just like a value in the configuration file.
\end{funcdesc}

\begin{funcdesc}{loadConfigFile}{schema, file\optional{,
                                 url\optional{, overrides}}}
  Load and return a configuration from an opened file object.  If
  \var{url} is omitted, one will be computed based on the
  \member{name} attribute of \var{file}, if it exists.  If no URL can
  be determined, all \keyword{\%include} statements in the
  configuration must use absolute URLs.  \var{schema} is a reference
  to a schema loaded by \function{loadSchema()} or
  \function{loadSchemaFile()}.
  The return value is a tuple containing the configuration object and
  a composite handler that, when called with a name-to-handler
  mapping, calls all the handlers for the configuration.
  The \var{overrides} argument is the same as for the
  \function{loadConfig()} function.
\end{funcdesc}

\begin{funcdesc}{loadSchema}{url}
  Load a schema definition from the URL \var{url}.
  \var{url} may be a URL, absolute pathname, or relative pathname.
  Fragment identifiers are not supported.
  The resulting
  schema object can be passed to \function{loadConfig()} or
  \function{loadConfigFile()}.  The schema object may be used as many
  times as needed.
\end{funcdesc}

\begin{funcdesc}{loadSchemaFile}{file\optional{, url}}
  Load a schema definition from the open file object \var{file}.  If
  \var{url} is given and not \code{None}, it should be the URL of
  resource represented by \var{file}.  If \var{url} is omitted or
  \code{None}, a URL may be computed from the \member{name} attribute
  of \var{file}, if present.  The resulting schema object can
  be passed to \function{loadConfig()} or \function{loadConfigFile()}.
  The schema object may be used as many times as needed.
\end{funcdesc}

The following exceptions are defined by this package:

\begin{excdesc}{ConfigurationError}
  Base class for exceptions specific to the \module{ZConfig} package.
  All instances provide a \member{message} attribute that describes
  the specific error, and a \member{url} attribute that gives the URL
  of the resource the error was located in, or \constant{None}.
\end{excdesc}

\begin{excdesc}{ConfigurationSyntaxError}
  Exception raised when a configuration source does not conform to the
  allowed syntax.  In addition to the \member{message} and
  \member{url} attributes, exceptions of this type offer the
  \member{lineno} attribute, which provides the line number at which
  the error was detected.
\end{excdesc}

\begin{excdesc}{DataConversionError}
  Raised when a data type conversion fails with
  \exception{ValueError}.  This exception is a subclass of both
  \exception{ConfigurationError} and \exception{ValueError}.  The
  \function{str()} of the exception provides the explanation from the
  original \exception{ValueError}, and the line number and URL of the
  value which provoked the error.  The following additional attributes
  are provided:

  \begin{tableii}{l|l}{member}{Attribute}{Value}
    \lineii{colno}
           {column number at which the value starts, or \code{None}}
    \lineii{exception}
           {the original \exception{ValueError} instance}
    \lineii{lineno}
           {line number on which the value starts}
    \lineii{message}
           {\function{str()} returned by the original \exception{ValueError}}
    \lineii{value}
           {original value passed to the conversion function}
    \lineii{url}
           {URL of the resource providing the value text}
  \end{tableii}
\end{excdesc}

\begin{excdesc}{SchemaError}
  Raised when a schema contains an error.  This exception type
  provides the attributes \member{url}, \member{lineno}, and
  \member{colno}, which provide the source URL, the line number, and
  the column number at which the error was detected.  These attributes
  may be \code{None} in some cases.
\end{excdesc}

\begin{excdesc}{SchemaResourceError}
  Raised when there's an error locating a resource required by the
  schema.  This is derived from \exception{SchemaError}.  Instances of
  this exception class add the attributes \member{filename},
  \member{package}, and \member{path}, which hold the filename
  searched for within the package being loaded, the name of the
  package, and the \code{__path__} attribute of the package itself (or
  \constant{None} if it isn't a package or could not be imported).
\end{excdesc}

\begin{excdesc}{SubstitutionReplacementError}
  Raised when the source text contains references to names which are
  not defined in \var{mapping}.  The attributes \member{source} and
  \member{name} provide the complete source text and the name
  (converted to lower case) for which no replacement is defined.
\end{excdesc}

\begin{excdesc}{SubstitutionSyntaxError}
  Raised when the source text contains syntactical errors.
\end{excdesc}


\subsection{Basic Usage}

The simplest use of \refmodule{ZConfig} is to load a configuration
based on a schema stored in a file.  This example loads a
configuration file specified on the command line using a schema in the
same directory as the script:

\begin{verbatim}
import os
import sys
import ZConfig

try:
    myfile = __file__
except NameError:
    myfile = os.path.realpath(sys.argv[0])

mydir = os.path.dirname(myfile)

schema = ZConfig.loadSchema(os.path.join(mydir, 'schema.xml'))
conf, handler = ZConfig.loadConfig(schema, sys.argv[1])
\end{verbatim}

If the schema file contained this schema:

\begin{verbatim}
<schema>
  <key name='server' required='yes'/>
  <key name='attempts' datatype='integer' default='5'/>
</schema>
\end{verbatim}

and the file specified on the command line contained this text:

\begin{verbatim}
# sample configuration

server www.example.com
\end{verbatim}

then the configuration object \code{conf} loaded above would have two
attributes:

\begin{tableii}{l|l}{member}{Attribute}{Value}
  \lineii{server}{\code{'www.example.com'}}
  \lineii{attempts}{\code{5}}
\end{tableii}


\section{\module{ZConfig.datatypes} --- Default data type registry}

\declaremodule{}{ZConfig.datatypes}
\modulesynopsis{Default implementation of a data type registry}

The \module{ZConfig.datatypes} module provides the implementation of
the default data type registry and all the standard data types
supported by \module{ZConfig}.  A number of convenience classes are
also provided to assist in the creation of additional data types.

A \dfn{datatype registry} is an object that provides conversion
functions for data types.  The interface for a registry is fairly
simple.

A \dfn{conversion function} is any callable object that accepts a
single argument and returns a suitable value, or raises an exception
if the input value is not acceptable.  \exception{ValueError} is the
preferred exception for disallowed inputs, but any other exception
will be properly propagated.

\begin{classdesc}{Registry}{\optional{stock}}
  Implementation of a simple type registry.  If given, \var{stock}
  should be a mapping which defines the ``built-in'' data types for
  the registry; if omitted or \code{None}, the standard set of data
  types is used (see section~\ref{standard-datatypes}, ``Standard
  \module{ZConfig} Datatypes'').
\end{classdesc}

\class{Registry} objects have the following methods:

\begin{methoddesc}{get}{name}
  Return the type conversion routine for \var{name}.  If the
  conversion function cannot be found, an (unspecified) exception is
  raised.  If the name is not provided in the stock set of data types
  by this registry and has not otherwise been registered, this method
  uses the \method{search()} method to load the conversion function.
  This is the only method the rest of \module{ZConfig} requires.
\end{methoddesc}

\begin{methoddesc}{register}{name, conversion}
  Register the data type name \var{name} to use the conversion
  function \var{conversion}.  If \var{name} is already registered or
  provided as a stock data type, \exception{ValueError} is raised
  (this includes the case when \var{name} was found using the
  \method{search()} method).
\end{methoddesc}

\begin{methoddesc}{search}{name}
  This is a helper method for the default implementation of the
  \method{get()} method.  If \var{name} is a Python dotted-name, this
  method loads the value for the name by dynamically importing the
  containing module and extracting the value of the name.  The name
  must refer to a usable conversion function.
\end{methoddesc}


The following classes are provided to define conversion functions:

\begin{classdesc}{MemoizedConversion}{conversion}
  Simple memoization for potentially expensive conversions.  This
  conversion helper caches each successful conversion for re-use at a
  later time; failed conversions are not cached in any way, since it
  is difficult to raise a meaningful exception providing information
  about the specific failure.
\end{classdesc}

\begin{classdesc}{RangeCheckedConversion}{conversion\optional{,
                                          min\optional{, max}}}
  Helper that performs range checks on the result of another
  conversion.  Values passed to instances of this conversion are
  converted using \var{conversion} and then range checked.  \var{min}
  and \var{max}, if given and not \code{None}, are the inclusive
  endpoints of the allowed range.  Values returned by \var{conversion}
  which lay outside the range described by \var{min} and \var{max}
  cause \exception{ValueError} to be raised.
\end{classdesc}

\begin{classdesc}{RegularExpressionConversion}{regex}
  Conversion that checks that the input matches the regular expression
  \var{regex}.  If it matches, returns the input, otherwise raises
  \exception{ValueError}.
\end{classdesc}


\section{\module{ZConfig.loader} --- Resource loading support}

\declaremodule{}{ZConfig.loader}
\modulesynopsis{Support classes for resource loading}

This module provides some helper classes used by the primary APIs
exported by the \module{ZConfig} package.  These classes may be useful
for some applications, especially applications that want to use a
non-default data type registry.

\begin{classdesc}{Resource}{file, url\optional{, fragment}}
  Object that allows an open file object and a URL to be bound
  together to ease handling.  Instances have the attributes
  \member{file}, \member{url}, and \member{fragment} which store the
  constructor arguments.  These objects also have a \method{close()}
  method which will call \method{close()} on \var{file}, then set the
  \member{file} attribute to \code{None} and the \member{closed} to
  \constant{True}.
\end{classdesc}

\begin{classdesc}{BaseLoader}{}
  Base class for loader objects.  This should not be instantiated
  directly, as the \method{loadResource()} method must be overridden
  for the instance to be used via the public API.
\end{classdesc}

\begin{classdesc}{ConfigLoader}{schema}
  Loader for configuration files.  Each configuration file must
  conform to the schema \var{schema}.  The \method{load*()} methods
  return a tuple consisting of the configuration object and a
  composite handler.
\end{classdesc}

\begin{classdesc}{SchemaLoader}{\optional{registry}}
  Loader that loads schema instances.  All schema loaded by a
  \class{SchemaLoader} will use the same data type registry.  If
  \var{registry} is provided and not \code{None}, it will be used,
  otherwise an instance of \class{ZConfig.datatypes.Registry} will be
  used.
\end{classdesc}


\subsection{Loader Objects}

Loader objects provide a general public interface, an interface which
subclasses must implement, and some utility methods.

The following methods provide the public interface:

\begin{methoddesc}[loader]{loadURL}{url}
  Open and load a resource specified by the URL \var{url}.
  This method uses the \method{loadResource()} method to perform the
  actual load, and returns whatever that method returns.
\end{methoddesc}

\begin{methoddesc}[loader]{loadFile}{file\optional{, url}}
  Load from an open file object, \var{file}.  If given and not
  \code{None}, \var{url} should be the URL of the resource represented
  by \var{file}.  If omitted or \code{None}, the \member{name}
  attribute of \var{file} is used to compute a \code{file:} URL, if
  present.
  This method uses the \method{loadResource()} method to perform the
  actual load, and returns whatever that method returns.
\end{methoddesc}

The following method must be overridden by subclasses:

\begin{methoddesc}[loader]{loadResource}{resource}
  Subclasses of \class{BaseLoader} must implement this method to
  actually load the resource and return the appropriate
  application-level object.
\end{methoddesc}

The following methods can be used as utilities:

\begin{methoddesc}[loader]{isPath}{s}
  Return true if \var{s} should be considered a filesystem path rather
  than a URL.
\end{methoddesc}

\begin{methoddesc}[loader]{normalizeURL}{url-or-path}
  Return a URL for \var{url-or-path}.  If \var{url-or-path} refers to
  an existing file, the corresponding \code{file:} URL is returned.
  Otherwise \var{url-or-path} is checked for sanity: if it
  does not have a schema, \exception{ValueError} is raised, and if it
  does have a fragment identifier, \exception{ConfigurationError} is
  raised.
  This uses \method{isPath()} to determine whether \var{url-or-path}
  is a URL of a filesystem path.
\end{methoddesc}

\begin{methoddesc}[loader]{openResource}{url}
  Returns a resource object that represents the URL \var{url}.  The
  URL is opened using the \function{urllib2.urlopen()} function, and
  the returned resource object is created using
  \method{createResource()}.  If the URL cannot be opened,
  \exception{ConfigurationError} is raised.
\end{methoddesc}

\begin{methoddesc}[loader]{createResource}{file, url}
  Returns a resource object for an open file and URL, given as
  \var{file} and \var{url}, respectively.  This may be overridden by a
  subclass if an alternate resource implementation is desired.
\end{methoddesc}


\section{\module{ZConfig.cmdline} --- Command-line override support}

\declaremodule{}{ZConfig.cmdline}
\modulesynopsis{Support for command-line overrides for configuration
                settings.}

This module exports an extended version of the \class{ConfigLoader}
class from the \refmodule{ZConfig.loader} module.  This provides
support for overriding specific settings from the configuration file
from the command line, without requiring the application to provide
specific options for everything the configuration file can include.

\begin{classdesc}{ExtendedConfigLoader}{schema}
  Construct a \class{ConfigLoader} subclass that adds support for
  command-line overrides.
\end{classdesc}

The following additional method is provided, and is the only way to
provide position information to associate with command-line
parameters:

\begin{methoddesc}{addOption}{spec\optional{, pos}}
  Add a single value to the list of overridden values.  The \var{spec}
  argument is a value specified, as described for the
  \function{\refmodule{ZConfig}.loadConfig()} function.  A source
  position for the specifier may be given as \var{pos}.  If \var{pos}
  is specified and not \code{None}, it must be a sequence of three
  values.  The first is the URL of the source (or some other
  identifying string).  The second and third are the line number and
  column of the setting.  These position information is only used to
  construct a \exception{DataConversionError} when data conversion
  fails.
\end{methoddesc}


\section{\module{ZConfig.substitution} --- String substitution}

\declaremodule{}{ZConfig.substitution}
\modulesynopsis{Shell-style string substitution helper.}

This module provides a basic substitution facility similar to that
found in the Bourne shell (\program{sh} on most \UNIX{} platforms).  

The replacements supported by this module include:

\begin{tableiii}{l|l|c}{code}{Source}{Replacement}{Notes}
  \lineiii{\$\$}{\code{\$}}{(1)}
  \lineiii{\$\var{name}}{The result of looking up \var{name}}{(2)}
  \lineiii{\$\{\var{name}\}}{The result of looking up \var{name}}{}
\end{tableiii}

\noindent
Notes:
\begin{description}
  \item[(1)]  This is different from the Bourne shell, which uses
              \code{\textbackslash\$} to generate a \character{\$} in
              the result text.  This difference avoids having as many
              special characters in the syntax.

  \item[(2)]  Any character which immediately follows \var{name} may
              not be a valid character in a name.
\end{description}

In each case, \var{name} is a non-empty sequence of alphanumeric and
underscore characters not starting with a digit.  If there is not a
replacement for \var{name}, the exception
\exception{SubstitutionReplacementError} is raised.
Note that the lookup is expected to be case-insensitive; this module
will always use a lower-case version of the name to perform the query.

This module provides these functions:

\begin{funcdesc}{substitute}{s, mapping}
  Substitute values from \var{mapping} into \var{s}.  \var{mapping}
  can be a \class{dict} or any type that supports the \method{get()}
  method of the mapping protocol.  Replacement
  values are copied into the result without further interpretation.
  Raises \exception{SubstitutionSyntaxError} if there are malformed
  constructs in \var{s}.
\end{funcdesc}

\begin{funcdesc}{isname}{s}
  Returns \constant{True} if \var{s} is a valid name for a substitution
  text, otherwise returns \constant{False}.
\end{funcdesc}


\subsection{Examples}

\begin{verbatim}
>>> from ZConfig.substitution import substitute
>>> d = {'name': 'value',
...      'top': '$middle',
...      'middle' : 'bottom'}
>>>
>>> substitute('$name', d)
'value'
>>> substitute('$top', d)
'$middle'
\end{verbatim}


\appendix
\section{Schema Document Type Definition \label{schema-dtd}}

The following is the XML Document Type Definition for \module{ZConfig}
schema:

\verbatiminput{schema.dtd}

\end{document}
