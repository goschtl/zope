\documentclass[12pt,english]{scrbook}
\usepackage{babel}
\usepackage[latin1]{inputenc}
\usepackage{url}
\usepackage{tabularx}
\usepackage{longtable}
\usepackage{graphicx}
\usepackage{booktabs}
\usepackage{rotating}
\usepackage{varioref}
\usepackage[colorlinks=true,linkcolor=blue,urlcolor=blue]{hyperref}


% 90 degrees rotated
\newcolumntype{R}{%
  >{\begin{turn}{90}%
          \hspace{0pt}}l%
  <{\end{turn}}%
}
\newcommand{\oh}{$\bullet$}


\hypersetup{
pdftitle={Zope X3 Security Target for EAL 1 (Draft)},
pdfauthor={Christian Theune {\textless}ct@gocept.com{\textgreater};Steve Alexander {\textless}steve@catbox.net{\textgreater};Jim Fulton {\textless}jim@zope.com{\textgreater}}
}


\subject{Zope X3}
\title{Security Target for EAL 1 (Draft)}
\author{Christian Theune \\
  Steve Alexander \\
  Jim Fulton \\
  Christian Zagrodnick}

\uppertitleback{
\begin{description}
    \item[Version:] \$Rev$ $\$ (Draft)
    \item[Date:] \$Date$ $\$
    \item[Author:] Christian Theune, ct@gocept.com
    \item[Author:] Steve Alexander, steve@catbox.net
    \item[Author:] Jim Fulton, jim@zope.com
    \item[Author:] Christian Zagrodnick, cz@gocept.com
    \item[DocumentID:] \$Id$ $\$
  \end{description}
}
\date{\today}



\begin{document}
\maketitle

%___________________________________________________________________________


\tableofcontents
\newpage
\listoftables


%___________________________________________________________________________



\chapter{ST introduction}


%___________________________________________________________________________



\section{ST identification}

\begin{description}
  
  \item [Document Title:] Zope X3, Security target

  \item [Document ID:]
    \$Id$ $\$
  
  \item [Document Version:] \$Rev$ $\$
  
  \item [Origin:] Zope Corporation public Subversion server
  
  \item [TOE Reference:] Zope X3 3.1/CC       
  % XXX still to define. Possible alternative: Zope CC 3.1

  \item [TOE Commercial Name:] Zope X3   
  % XXX to define, depending on TOE Reference

  \item [TOE Short Description:] A platform independent web application server
  and framework written in Python


  \item [Product Type:] Web Application Server


  \item [Evaluation Body:] Evaluation Body of T"UV Informationstechnik GmbH,
  Germany

  \item [Certification Body:] Certification Body of T"UV Informationstechnik
  GmbH, Germany


\end{description}

This ST is based upon Common Criteria, Version 2.1 
The TOE consists of the following component:

\begin{longtable}[c]{lll}
  \toprule
  Component & Version & Supplier \\
  \midrule \endhead
  Zope & X3 & Zope Corporation \\
  % The version needs to be defined 
  \bottomrule
  \caption{TOE Components}
\end{longtable}


%___________________________________________________________________________

\section{ST overview}

Zope 3 is a general purpose web application server written in Python. 

Beeing an application server Zope 3 provides developers with a flexible
security machinery that allows complex applications developed for Zope 3 to be
effectively protected.

This includes modules for identification, authentication and authorization and
integrates seemlessly with all other subsystems Zope 3 provides.

Developers using Zope 3 instruct the security machinery to protect objects and
operations in the application server with given permissions.

Administrators managing a Zope 3 based application grant users permissions to
use certain objects and operations and configure the server and application to
conform to local security policies.

The flexibility of the system allows tailored use of the security functions on
multiple levels, to allow easy integration of third party Zope 3 applications
into an existing IT environment.

%___________________________________________________________________________



\section{ISO/IEC 15408 (CC) Conformance}

This ST is claimed to be conforming with the ISO/IEC 15408:1999 (Common
Criteria, Version 2.1 with final interpretations) and its following
parts:

\begin{itemize}
  
  \item Part 2 and
  
  \item Part 3, EAL1.

\end{itemize}

The assurance level is EAL 1.


%___________________________________________________________________________



\chapter{TOE description}


%___________________________________________________________________________



\section{Overview}

Zope 3 (also referred to as ``Zope'') is a component based framework that may
be used to build web applications. It's development is historically focused,
but not limited, on building content management systems.

It is written as platform independent software using the Python programming
language. Therefore it is available for Windows NT, Linux, MacOS X and other
operating systems.

Zope 3 features a set of core components that form an infrastructure for 
building applications by reusing existing components and adding new components
that work together by defined interfaces.

The core functionality contains a web server with WebDAV support, a ftp server
and a XML/RPC server.  It has components that provide functionality for
security management including administration of users, roles and permissions.
Other core components cover an object database, indexing mechanisms,
workflow, a web interface, SQL support, an XML-based and a non-XML based templating
mechanism, Python Scripting, I18n and L10n support and many more.

Finally Zope can be extended by the use of packages that can contain
configuration directives, templates, Python code and classes. Those are
intended to work together seamlessly using the Component Architecture to plug
them together into more complex systems.


%___________________________________________________________________________



\section{TOE definition}

As a general rule it is possible to describe all activities with and within Zope as
``operations'' performed on ``objects''. The need for security adds a protecting
subject to this by guarding operations with ``permissions''.

Users of the system are internally identified with ``principals'' which act on
their behalf.  Those principals are granted permissions (both statically via
configuration files and dynamically via settings in the object database) within
an context to allow them performing operations on a selected set of objects.

Principals are authenticated in various ways depending on the means of
connection to a server.  Authentication usually envolves a username-password
pair such as for FTP-Authentication and HTTP-Basic-Authentication.  Other
authentication mechanisms are possible.


%___________________________________________________________________________



\section{TOE Development and Production}

The development of Zope 3 is driven by the Zope Corporation together with the
free community of Zope developers. The Zope 3 source code is free to be
retrieved and used by everybody.

The official Zope 3 source code is maintained within a centralized source-code
control system.  Everybody is free to retrieve any available version of the
code anonymously. The certified version is available on a named branch and
identified by a tag.

To ensure a stable production every developer wishing to directly access the
repository must retrieve authorisation from Zope Corporation. This is
expressed by providing a signed contributors agreement,
\url{http://dev.zope.org/DevHome/Subversion/Contributor.pdf}.

Write access to the repository is only available through ssh and by providing
a public key to the maintainer of the repository.

All changes to source code and other files in the repository are reported
publically to interested persons including those persons that are responsible
for overseeing the quality and direction of parts of Zope.

Any change to a file in the repository causes that file to have a new version
number and the exact change is recorded. Before writing a change back to the
repository on a branch that is declared for public use or for main development
(release branches, HEAD, main development branches) the developer must
successfully run the prepared unit tests to assure that no code breaks when
applying his changes. Additionally every developer is required to provide unit
test for new code he writes or problems he solves, as far as applicable.


%___________________________________________________________________________



\section{TOE Life Cycle}

The TOE is developed in cycles. New features are introduced in iterative steps
called ``feature release'' and solutions to known problems are introduced by
steps called ``bugfix releases''.

The version numbers of the TOE releases express if it is a feature or bugfix
release like this: 3.f.b where f and b are continuous given numbers and f
expresses the f-th feature release for Zope 3 and b expresses the b-th bugfix
release for the f-th feature release. Every feature release starts with bugfix
release 0 in which case the bugfix number may be omitted. (E.g. 3.1.4,
3.1.0/3.1, 3.0.0/3.0)

Test releases are identified by adding their grade (a for alpha, b for beta,
  rc for release candidate) at the end of the version number that it is
targeted at.  (3.1.4b2 would be the second beta release for the upcoming
  version 3.1.4)

New features are proposed and agreed within the developers community by the use
of mailing lists and wiki pages. They are incorporated in an agreed feature
release.

Until agreed to be ready for public test the development and until all
features are available (but maybe untested), development of a feature release
happens on the SVN trunk (a special branch, also known as HEAD). When starting
public releases, no further features are allowed to be introduced and the
development enters maintenance mode. Therefore a named branch is created to
identify changes that are applied for maintenance.  New features will be
introduced on the trunk that is heading for the next feature release.

Therefore an overall of about 3 concurrent maintained versions can exist:

\begin{itemize}
  
  \item old feature release in maintenance mode

  \item upcoming feature release, already in maintenance mode but not stable

  \item upcoming feature release in free development mode

\end{itemize}


%___________________________________________________________________________



\section{TOE Boundaries}


%___________________________________________________________________________



\subsection{Physical Boundaries}

The TOE is physically limited by the files that are included in a Zope 3
source software distribution. A binary distribution may include more software
packages that are not part of the TOE. (E.g. Python runtime libraries)


%___________________________________________________________________________



\subsection{TOE Logical Boundaries}

The logical boundary for the TOE consists of several security-relevant
sub-systems of Zope 3:

\begin{itemize}

  \item Protection

  \item Authentication

  \item Authorization / Access Control

  \item Auditing

  \item Transaction Management

  \item Undo

  \item Publication / Server

\end{itemize}

See section \vref{toe-sec-funcs} for more details regarding those sub-systems.

%___________________________________________________________________________



\chapter{TOE security environment}


%___________________________________________________________________________



\section{Assets}

The following primary assets have been identified:

\begin{longtable}[c]{lp{10cm}}
  \toprule 
  Asset Name & Description \\
  \midrule\endhead

  (Content) Objects & 
  Generic objects (instances of Python classes) that
  are stored and controlled by Zope and carry
  information that is to be protected. Objects are
  stored in a connected manner that is typically
  hierarchical and allows the derivation of
  information by the objects context. \\
  \bottomrule
  \caption{Primary Assets}
  \label{tab-assets}
\end{longtable}

The following secondary assets have been identified:

\begin{longtable}[c]{lp{10cm}}
  \toprule 
  Asset Name & Description \\
  \midrule\endhead
  Host System
   & 
  The unit of computer hardware and software that
  forms the environment of Zope to run on. (E.g.
  a PC server with Windows 2000 or Linux installed)
   \\

  Operations
   & 
  Operations are the way of accessing and modifying
  data provided by (content) objects.
   \\

  Principals
   & 
  Principals are the systems representation of acting
  individuals. A principal acts in behalf of the user
  and represents a (content) object of it's own.
   \\

  Permission
   & 
  A permission is a name guarding an operation.
   \\

  Permission grants
   & 
  A permission grant associates a principal with a
  permission to allow or deny an operation in the context.
  As a third state, permissions may be declared to
  be acquired from the context.
   \\

  Audit data
   & 
  The data generated by the TOE audit subsystem.
   \\

  Transaction data
   & 
  All operations within Zope are held within ACID
  compatible transactions that are bound to each
  request from the outside and associated with a
  principal.
  \\
  \bottomrule
  \caption{Secondary Assets}
  \label{tab-sec-assets}

\end{longtable}
%___________________________________________________________________________



\section{Subject}

Zope has a concept of interactions, which model the interaction of one
or more users with the system.  An interaction keeps track of the
users that are participating in the interaction as ``participations''.
In the TOE, interactions will have single users participating through
server request (for example, Web requests).  Interactions are referred
to as ``subjects'' in the TOE.


%___________________________________________________________________________



\section{Operations}

Operations are performed on objects. They are defined in an objects class. A
class is defined in the Python programming language and is identified by a
fully qualified name.

An operation is a name defined in a class. It may take a form of an attribute,
a method or some other related python thing.

There are two possible kinds of access to an operation: Reading such as
reading an attribute or calling a method. Writing such as setting or deleting
an attribute. Reading and writing can be guarded with different permission
grants.


%___________________________________________________________________________



\section{Assumptions (about the environment)}

The following assumptions need to be made about the TOE environment:

\begin{longtable}[c]{lp{10cm}}
  \toprule
  Assumption Name & Description \\
  \midrule

  A.OS & 
  The machine and the operating system Zope is running on is physically secure.
  The system is administrated such that the system is free from malicious
  software like viruses and Trojan horses.  \\

  A.Admin & 
  The ``system-administrator'' of the above
  mentioned machine is trustworthy.
   \\

  A.Network & 
  A network connection to the Zope services is
  present. All other network connection are
  secure in such a way that the integrity of
  the machine and operating system is preserved.
   \\

  A.Client & 
  The connection between client and Zope server is
  secure in a sense that the identification and
  authentication data is not monitored or interfered.
   \\

  A.Credential & 
  The user is keeping the credential to authenticate
  secret. \\

  \bottomrule
  \caption{Assumptions about the TOE environment.}
  \label{tab-A}
\end{longtable}

%___________________________________________________________________________



\section{Threats}

The following threat agents have been identified:

\begin{itemize} 
  
  \item Users having correct authentication credentials who might try to
  acquire more permission grants to get access to operations they should not.

  \item Users without correct authentication credentials for a certain
  principal trying to authenticate as this.

\end{itemize}

Specific threat agents with specific motivation, resources and skills have to
be identified for any specific application build on Zope 3. From the point of a
generic application server, attackers are either to be expected to  be
authenticated and trying to extend their level of access or not having been
authenticated at all and trying to break into the system.

The following threats against the assets have been identified:

\begin{longtable}[c]{lp{6cm}p{4cm}}
  \toprule
  Threat & Description & Asset\\
  \midrule\endhead

  T.IA
   & 
  An attacker might impersonate an authorized
  principal without providing the necessary
  credentials.
   & 
  Principal
   \\
  

  T.Perm
   & 
  A principal changes the permission grants
  without having the right to do so.
   & 
  Permission grants
   \\
  

  T.Operation
   & 
  A principal performs an operation on an object
  without having the correct permission.
   & 
  Operation, Object
   \\
  

  T.AuditFake
   & 
  An attacker might convince the audit data
  generation functions to log false information
  (date, time, type of event, outcome, user)
   & 
  Audit data
   \\
  

  T.RIP
   & 
  An attacker might try to make the system use
  residual information for deciding to allow
  or deny access to an operation to gain more
  access than intended.
   & 
  Secondary assets
   \\
  

  T.Transaction
   & 
  An attacker might try to perform commit or
  abort operations on foreign transactions to
  perform operations on the behalf of other
  users.
   & 
  All assets in ZODB
   \\
  

  T.Undo
   & 
  An attacker might try to perform an Undo
  operation to invalid revisions.
   & 
  All assets in ZODB
   \\
  

  T.Timestamps
   & 
  An attacker might try to hide his actions
  by making the system create false timestamps
  which would result in wrong association to a
  user on dynamic IP address ranges.
   & 
  Audit data
   \\
  

  T.TrustedPath
   & 
  An attacker might try to use ``user data import''
  or ``user data export'' without being a local
  user and using the trusted path.
   & 
  Object
   \\
  

  T.Host
   & 
  An attacker might use Python functions that
  result in direct access to the host environment
  therefore compromising the host and Zope itself.
   & 
  Host, Object
  \\
  \bottomrule
\caption{Threats Against Assets}
\label{tab-threats}
\end{longtable}
  


%___________________________________________________________________________



\section{Organisational security policies}

OSPs are to be defined by the developer who creates applications using Zope and
the customer running those applications.  Zope as a general purpose application
server does neither require nor impose any OSPs.


%___________________________________________________________________________



\chapter{Security objectives}


%___________________________________________________________________________



\section{Security objectives for the TOE}

The following security objectives have been defined for the TOE:

\begin{longtable}[c]{lp{10cm}}
  \toprule
  Objective Name & Description \\
  \midrule\endhead
  
  O.IA
   & 
  All principals must be accurately identified and
  authenticated with the exception of the ``unauthenticated''
  principal.
   \\

  O.Delegation
   & 
  Provide the ability to securely delegate control. Users can
  delegate the ability to control access to selected
  operations to others. To delegate a permission, a meta permission
  that allows you to delegate this permission must be granted.
   \\

  O.Audit
   & 
  The TOE will provide the means of recording any
  security relevant events, so as to assist an
  administrator in the detection of potential attacks
  or misconfiguration of the TOE security features
  that would leave the TOE susceptible to attack, and
  also to hold users accountable for any actions
  they perform that are relevant to security.
   \\

  O.Protect
   & 
  The TOE will protect itself against external
  interference or tampering by untrusted subjects or
  attempts by untrusted subjects to bypass the TOE
  security functions.
   \\

  O.Access
   & 
  The TOE ensures that access to objects is always
  mediated by operations and guarded by permissions.
   \\

  O.Integrity
   & 
  Whenever an unhandled error within the context of a
  running transaction occurs (related or unrelated
  to security) the transaction will be rolled back
  and the system will be in the state before the
  transaction started.
   \\

  O.Attributes &  All security attributes (e.g. principal or permission
    identifiers) together must form a meaningful whole at all times. \\

  O.ManageRisk
   & 
  Provide the ability to manage risk by trading off
  functionality against risk. For example, we can
  make it easier to access the system to perform
  operations whose potential negative impact is
  low, but make it more difficult to access the
  system in a way that allows operations with high
  negative impact.
  \\
  \bottomrule
  \caption{Security Objectives for the TOE}
\end{longtable}


%___________________________________________________________________________



\section{Security objectives for the environment}

The following security objectives have been defined for the TOE environment:

\begin{longtable}[c]{lp{10cm}}
  \toprule
  Assumption Name  &  Description \\
  \midrule\endhead

  OE.OS
   & 
  The machine and the operating system Zope is running
  on is physically secure.
   \\

  OE.Trust
   & 
  Those responsible for the TOE must be trustworthy.
   \\

  OE.Auditlog
   & 
  Administrators of the TOE must ensure that audit
  facilities are used and managed effectively. In
  particular:

  \begin{itemize}
  
    \item Appropriate action must be taken to ensure continued audit logging,
    e.g. by regular archiving of logs before audit trail exhaustion to ensure
    sufficient free space.

    \item Audit logs should be inspected on a regular basis, and appropriate
    action should be taken on the detection of breaches of security, or events
    that are likely to lead to a breach in the future.

  \end{itemize}
   \\

  OE.Network
   & 
  A network connection to the Zope services is present.
  All other network connections are secure in such a
  way that the integrity of the machine and operating
  system is preserved.
   \\

  OE.Client
   & 
  The connection between client and Zope server is secure
  in a sense that the identification and authentication
  data is not monitored or interfered.
   \\

  OE.Credential
   & 
  The user is keeping the credentials to authenticate
  secret.
  \\
  \bottomrule
  \caption{Security Objectives for the Environment}
\end{longtable}


%___________________________________________________________________________



\chapter{IT Security requirements}


%___________________________________________________________________________



\section{TOE security requirements}


%___________________________________________________________________________



\subsection{TOE security functional requirements}

The following functional requirements identify the TOE functional requirements.
They have been drawn from the CC Part 2 functional requirements components.


%___________________________________________________________________________



\subsubsection{Class FAU: Audit data generation}


%___________________________________________________________________________



\minisec{FAU{\_}GEN.1 Audit data generation}
\begin{description}
  
  \item[FAU\_GEN.1.1] The TSF shall be able to generate an audit record of the
  following auditable events:
  
\newcounter{listcnt2}
\begin{list}{\alph{listcnt2})}
{
\usecounter{listcnt2}
\setlength{\rightmargin}{\leftmargin}
}
\item Startup and shutdown of audit functions;

\item All auditable events for the \emph{\[minimum\]} level of audit.

\end{list}

%[depart_definition]
%[depart_definition_list_item]
%[visit_definition_list_item]
\item[FAU{\_}GEN.1.2]
%[visit_definition]

The TSF shall record within each audit record at least the
following information:
\newcounter{listcnt3}
\begin{list}{\alph{listcnt3})}
{
\usecounter{listcnt3}
\setlength{\rightmargin}{\leftmargin}
}
\item 
Date and time of the event, type of event, subject identity,
and the outcome (success or failure) of the event; and

\item \textbf{For each audit event type, based on auditable event definitions}
of the functional components included in the ST,
\emph{{[}assignment: the ID of the corresponding interaction]}

\end{list}

%[depart_definition]
%[depart_definition_list_item]
\end{description}


%___________________________________________________________________________



\minisec{FAU{\_}GEN.2 User identity assocation}
\begin{description}
%[visit_definition_list_item]
\item[FAU{\_}GEN.2.1]
%[visit_definition]

The TSF shall be able to associate each auditable event with the identity
of the user that caused the event.

%[depart_definition]
%[depart_definition_list_item]
\end{description}


%___________________________________________________________________________



\subsubsection{Class FDP: Data protection}


%___________________________________________________________________________



\minisec{FDP{\_}ACC.2 Complete access control}
\begin{description}
%[visit_definition_list_item]
\item[FDP{\_}ACC.2.1 ]
%[visit_definition]

The TSF shall enforce the \emph{\[formal security policy\]} on
\emph{\[subjects: interactions and objects: content objects\]} and all
operations among subjects and objects covered by the SFP.

\item[FDP{\_}ACC.2.2]

The TSF shall ensure that all operations between any
subject in the TSC and any object within the TSC are covered by an
access control SFP.

\end{description}


%___________________________________________________________________________



\minisec{FDP{\_}ACF.1 Security attribute based access control}
\begin{description}
%[visit_definition_list_item]
\item[FDP{\_}ACF.1.1]
%[visit_definition]

The TSF shall enforce the \emph{{[}formal security policy]} to objects
based on \emph{{[}the interaction principal, the permission required for
the operation and the grants or denials of the permission for that
object or it's ancestor objects]}.

%[depart_definition]
%[depart_definition_list_item]
%[visit_definition_list_item]
\item[FDP{\_}ACF.1.2]
%[visit_definition]

The TSF shall enforce the following rules to determine
if an operation among controlled subjects and controlled objects is
allowed:
\begin{itemize}
\item {} 
Access is denied if there is a denial for the subject's
principal for the required permission on the object.

\item {} 
Access is allowed if there is a grant and there is not a denial
for the subject's principal for the required permission on the object.

\item {} 
Access is denied if access would be denied for the subject's
principal for the required permission on the parent of the
object.

\item {} 
Access is allowed if access would be allowed and would not be
denied for the subject's principal for the required permission
on the parent of the object.

\end{itemize}

where the required permission is the permission required to
perform the desired operation on the object.

%[depart_definition]
%[depart_definition_list_item]
%[visit_definition_list_item]
\item[FDP{\_}ACF.1.3]
%[visit_definition]

The TSF shall explicitly authorise access of subjects to
objects based on the following additional rules: \emph{{[}none]}

%[depart_definition]
%[depart_definition_list_item]
%[visit_definition_list_item]
\item[FDP{\_}ACF.1.4]
%[visit_definition]

The TSF shall explicitly deny access of subjects to objects
based on the following additional rules: \emph{{[}none]}

%[depart_definition]
%[depart_definition_list_item]
\end{description}


%___________________________________________________________________________





%___________________________________________________________________________



\minisec{FDP{\_}ITC.1 Import of user data without security attributes}
\begin{description}
%[visit_definition_list_item]
\item[Note]
%[visit_definition]

The TOE may, initially, satisfy the requirements in this
function by not supporting data import, or, by only
supporting import from outside the TSC (outside the
software interfaces).

%[depart_definition]
%[depart_definition_list_item]
%[visit_definition_list_item]
\item[FDP{\_}ITC.1.1]
%[visit_definition]

The TSF shall enforce the \emph{{[}formal security policy]} when importing user 
data, controlled under the SFP, from outside of the TSC.

%[depart_definition]
%[depart_definition_list_item]
%[visit_definition_list_item]
\item[FDP{\_}ITC.1.2]
%[visit_definition]

The TSF shall ignore any security attributes associated with the user data 
when imported from outside the TSC.

%[depart_definition]
%[depart_definition_list_item]
%[visit_definition_list_item]
\item[FDP{\_}ITC.1.3]
%[visit_definition]

The TSF shall enforce the following rules when importing user data 
controlled under the SFP from outside the TSC:
\begin{quote}

No security attributes will be set when importing. The imported
user data will have no security attributes.
\begin{description}
%[visit_definition_list_item]
\item[Note]
%[visit_definition]

Access to the imported data will be governed by the grants in
the location the data are imported into, as described in
FDP{\_}ACF.1.2.

%[depart_definition]
%[depart_definition_list_item]
\end{description}
\end{quote}

%[depart_definition]
%[depart_definition_list_item]
\end{description}


%___________________________________________________________________________



\minisec{FDP{\_}ITC.2 Import of user data with security attributes}
\begin{description}
%[visit_definition_list_item]
\item[Note]
%[visit_definition]

The TOE may, initially, satisfy the requirements in this
function by not supporting data import, or, by only
supporting import from outside the TSC (outside the
software interfaces).

%[depart_definition]
%[depart_definition_list_item]
%[visit_definition_list_item]
\item[FDP{\_}ITC.2.1]
%[visit_definition]

The TSF shall enforce the \emph{{[}formal security policy]} when importing user 
data, controlled under the SFP, from outside of the TSC.

%[depart_definition]
%[depart_definition_list_item]
%[visit_definition_list_item]
\item[FDP{\_}ITC.2.2 ]
%[visit_definition]

The TSF shall use the security attributes associated with the imported 
user data.

%[depart_definition]
%[depart_definition_list_item]
%[visit_definition_list_item]
\item[FDP{\_}ITC.2.3]
%[visit_definition]

The TSF shall ensure that the protocol used provides for the unambiguous 
association between the security attributes and the user data received.

%[depart_definition]
%[depart_definition_list_item]
%[visit_definition_list_item]
\item[FDP{\_}ITC.2.4]
%[visit_definition]

The TSF shall ensure that interpretation of the security attributes of 
the imported user data is as intended by the source of the user data.

%[depart_definition]
%[depart_definition_list_item]
%[visit_definition_list_item]
\item[FDP{\_}ITC.2.5]
%[visit_definition]

The TSF shall enforce the following rules when importing user data 
controlled under the SFP from outside the TSC:
\begin{itemize}
\item {} 
If any imported data has security attributes that refer to
permissions or principals not defined in the target system, the
entire import will fail and an explanatory error will be provided.

\end{itemize}

%[depart_definition]
%[depart_definition_list_item]
\end{description}


%___________________________________________________________________________



\minisec{FDP{\_}RIP.1 Subset residual information protection}
\begin{description}
%[visit_definition_list_item]
\item[FDP{\_}RIP.2.1]
%[visit_definition]

The TSF shall ensure that any previous information content of a
resource is made unavailable upon the \emph{\[selection: deallocation of
security attributes from\]} all objects.
\end{description}

Note: This includes removing references to the security attributes beeing
deallocated. E.g. permission grants and denials belonging to a principal beeing
deallocated.

%___________________________________________________________________________



\minisec{FDP{\_}ROL.2{\_}TRANSACTIONS Advanced Rollback}
\begin{description}
%[visit_definition_list_item]
\item[FDP{\_}ROL.2.1 ]
%[visit_definition]

The TSF shall permit \emph{{[}the rollback of all
operations on all objects]}.

%[depart_definition]
%[depart_definition_list_item]
%[visit_definition_list_item]
\item[FDP{\_}ROL.2.2 ]
%[visit_definition]

The TSF shall permit operations to be rolled
back \emph{{[}at any time before the transaction in which the operation was
performed is committed]}.
\begin{description}
%[visit_definition_list_item]
\item[Note ]
%[visit_definition]

This statement may not apply to cached data created
during the course of a transaction.

%[depart_definition]
%[depart_definition_list_item]
\end{description}

%[depart_definition]
%[depart_definition_list_item]
\end{description}


%___________________________________________________________________________



\minisec{FDP{\_}ROL.1{\_}UNDO Basic rollback}
\begin{description}
%[visit_definition_list_item]
\item[FDP{\_}ROL.1.1 ]
%[visit_definition]

The TSF shall enforce the \emph{{[}formal security policy]} to permit
the rollback of the \emph{{[}operations that caused changes]} on the \emph{{[}content
objects]}.

%[depart_definition]
%[depart_definition_list_item]
%[visit_definition_list_item]
\item[FDP{\_}ROL.1.2 ]
%[visit_definition]

The TSF shall permit operations to be rolled back
within the \emph{{[}period of time for which the old revisions of the objects
exist]}.

%[depart_definition]
%[depart_definition_list_item]
\end{description}


%___________________________________________________________________________



\subsubsection{Class FIA: Identification and authentication}


%___________________________________________________________________________



\minisec{FIA{\_}AFL{\_}z.1 Authentication failure handling}
\begin{description}
%[visit_definition_list_item]
\item[FIA{\_}AFL{\_}z.1.1]
%[visit_definition]

The TSF shall detect when there are configurable number of consecutive
unsuccessful authentication attempts for a single login name,
with no intermediate successful attempts.

%[depart_definition]
%[depart_definition_list_item]
%[visit_definition_list_item]
\item[FIA{\_}AFL{\_}z.1.2 ]
%[visit_definition]

When the defined number of unsuccessful authentication attempts
has been surpassed, the TSF shall
\begin{itemize}
\item {} 
Disable authentication against the indicated login name for a
configurable period of time.

\end{itemize}

%[depart_definition]
%[depart_definition_list_item]
\end{description}


%___________________________________________________________________________



\minisec{FIA{\_}ATD.1 User attribute definition}
\begin{description}
%[visit_definition_list_item]
\item[FIA{\_}ATD.1.1 ]
%[visit_definition]

The TSF shall maintain the following list of security attributes
belonging to individual principals \emph{{[}uniqueid, credentials, grants
and denials]}

%[depart_definition]
%[depart_definition_list_item]
\end{description}


%___________________________________________________________________________



\minisec{FIA{\_}UAU.1 Timing of authentication}
\begin{description}
%[visit_definition_list_item]
\item[FIA{\_}UAU.1.1 ]
%[visit_definition]

The TSF shall allow \emph{{[}only those operations granted to the
unauthenticated principal]} on behalf of the user before the user is
authenticated.

\emph{{[}Note: It is possible to deny all operations to the anonymous
principal. This means that a user must login before any operations may
be performed on their behalf. This fullfills the terms of FIA{\_}UAU.2]}

%[depart_definition]
%[depart_definition_list_item]
%[visit_definition_list_item]
\item[FIA{\_}UAU.1.2 ]
%[visit_definition]

The TSF shall require each \emph{{[}user]} to be successfully
authenticated before allowing any other TSF-mediated actions on behalf
of that user.

%[depart_definition]
%[depart_definition_list_item]
\end{description}


%___________________________________________________________________________



\minisec{FIA{\_}UAU.5 Multiple authentication systems}
\begin{description}
%[visit_definition_list_item]
\item[FIA{\_}UAU.5.1 ]
%[visit_definition]

The TSF shall provide \emph{{[}extensible support for multiple
authentication mechanisms. The default mechanism uses login names
and passwords in combination with HTTP Basic Authentication and FTP
authentication]}

%[depart_definition]
%[depart_definition_list_item]
%[visit_definition_list_item]
\item[FIA{\_}UAU.5.2]
%[visit_definition]

The TSF shall authenticate any users claimed identity according to
the \emph{{[}system configuration, provided credentials, such as a
username/password pair and the protocol used]}

%[depart_definition]
%[depart_definition_list_item]
\end{description}


%___________________________________________________________________________



\minisec{FIA{\_}UAU.6 Re-authentication}
\begin{description}
%[visit_definition_list_item]
\item[FIA{\_}UAU.6.1 ]
%[visit_definition]

The TSF shall re-authenticate the user under the conditions
\begin{itemize}
\item {} 
If the credentials held by the user agent have expired due to
a configurable time limit.

\item {} 
If the authenticated user does not have the required permissions to
perform a requested operation but the presentation of different
credentials might associate him with a principal that holds enough
permission grants to perform the requested operation.

\end{itemize}

%[depart_definition]
%[depart_definition_list_item]
\end{description}


\minisec{FIA{\_}UID.1 Timing of identification}
\begin{description}
%[visit_definition_list_item]
\item[FIA{\_}UID.1.1 ]
%[visit_definition]

The TSF shall allow \emph{\[only those operations granted to the
unauthenticated principal\]} on behalf of the user before the user is
identified.

\emph{Note:} It is possible to deny all operations to the anonymous
principal. This means that a user must login before any operations may
be performed on their behalf. This fullfills the terms of FIA\_UID.2

%[depart_definition]
%[depart_definition_list_item]
%[visit_definition_list_item]
\item[FIA{\_}UID.1.2 ]
%[visit_definition]

The TSF shall require each user to be successfully
identified before allowing any other TSF-mediated actions on behalf
of that user.

%[depart_definition]
%[depart_definition_list_item]
\end{description}
%___________________________________________________________________________



\minisec{FIA{\_}USB.1 User-subject binding}
\begin{description}
%[visit_definition_list_item]
\item[FIA{\_}USB.1.1]
%[visit_definition]

The TSF shall associate the appropriate user security
attributes with subjects acting on behalf of that user.

%[depart_definition]
%[depart_definition_list_item]
\end{description}


%___________________________________________________________________________



\subsubsection{Class FMT: Security management}


%___________________________________________________________________________



\minisec{FMT{\_}MOF.1 Management of security functions}
\begin{description}
%[visit_definition_list_item]
\item[FMT{\_}MOF.1.1]
%[visit_definition]

The TSF shall restrict the ability to \emph{{[}determine the
behaviour of, disable, enable or modify the behaviour of]} the
\emph{{[}authentication]} functions to \emph{{[}assignment: 
authorized administrators]}.

%[depart_definition]
%[depart_definition_list_item]
%[visit_definition_list_item]
\item[Note]
%[visit_definition]

This includes for example adding and removing principals (for example,
users) and changing the authentication schemes. Those actions can be
protected by different permissions.

%[depart_definition]
%[depart_definition_list_item]
\end{description}


%___________________________________________________________________________




\minisec{FMT{\_}MSA.1 Management of security attributes}
\begin{description}
\item[FMT{\_}MSA.1.1.grants]
    The TSF shall enforce the \emph{{[}formal security policy]} to restrict the
    ability to \emph{{[}query, modify, delete, and add]} the security
    attributes \emph{{[}permission grants and denials]} to \emph{{[}authorized
    grantors]}.

\item[FMT{\_}MSA.1.2.loginname]
    The TSF shall enforce the \emph{{[}formal security policy]} to restrict the
    ability to \emph{{[}query and modify]} the security attribute
    \emph{{[}login name]} to \emph{{[}authorized administrators and users
    authorized to modify their own authentication data]}.

\item[FMT{\_}MSA.1.3.password]
    The TSF shall enforce the \emph{{[}formal security policy]} to restrict
    the ability to \emph{{[}modify]} the security attribute
    \emph{{[}password]} to \emph{{[}authorized administrators and users authorized to
    modify their own authentication data]}.

\end{description}

\minisec{FMT{\_}MSA.2 Secure security attributes}

\begin{description}

\item[FMT{\_}MSA.2.1]

    The TSF shall ensure that only secure values are accepted for security
    attributes.

\end{description}

%___________________________________________________________________________



\minisec{FMT{\_}MSA.3 Static attribute initialisation}
\begin{description}
%[visit_definition_list_item]
\item[FMT{\_}MSA.3.1]
%[visit_definition]

The TSF shall enforce the \emph{{[}formal security policy]} to provide 
\emph{{[}restrictive]} default values for security attributes that are used to 
enforce the SFP.

%[depart_definition]
%[depart_definition_list_item]
%[visit_definition_list_item]
\item[FMT{\_}MSA.3.2 ]
%[visit_definition]

The TSF shall allow \emph{{[}no one]} to specify alternative
initial values to override the default values when an object or
information is created.

%[depart_definition]
%[depart_definition_list_item]
%[visit_definition_list_item]
\item[Note]
%[visit_definition]

Security attributes are expressed as collections of grants or
denials. The default is an empty collection.

%[depart_definition]
%[depart_definition_list_item]
\end{description}


%___________________________________________________________________________



\minisec{FMT{\_}SMR.1 Security roles}

\begin{description}
    \item[FMT{\_}SMR.1.1]

    The TSF shall maintain the roles:
\begin{description}
\item[Authorized administrator]

Users who can perform system-wide security functions. These are
people who have the zope.ManageSecurity permission.

\item[Grantor ]

Users who have the ability to grant or deny permissions to
users for objects.  These are users who have any of the grant
meta-permissions.

\item[Users authorized to modify their own authentication data]

The role name says it all.

\end{description}

\item[FMT{\_}SMR.1.2]

The TSF shall be able to associate \emph{\[principals\]} with roles.

\end{description}


%___________________________________________________________________________



\subsubsection{Class FPT: Protection of the TSF}


%___________________________________________________________________________



\minisec{FPT{\_}AMT.1 Abstract machine testing}
\begin{description}
%[visit_definition_list_item]
\item[FPT{\_}AMT.1.1 ]
%[visit_definition]

The TSF shall run a suite of tests \emph{{[}as an offline
operation]} to demonstrate the correct operation of the security
assumptions provided by the abstract machine that underlies the
TSF.

%[depart_definition]
%[depart_definition_list_item]
\end{description}


%___________________________________________________________________________



\minisec{FPT{\_}FLS.1 Failure with preservation of secure state}
\begin{description}
%[visit_definition_list_item]
\item[FPT{\_}FLS.1.1 ]
%[visit_definition]

The TSF shall preserve a secure state when the following types of
failures occur: \emph{{[}process termination, resource
exhaustion and host restart]}.

%[depart_definition]
%[depart_definition_list_item]
\end{description}


%___________________________________________________________________________



\minisec{FPT{\_}RVM.1 Non-bypassability of the TSP}
\begin{description}
%[visit_definition_list_item]
\item[FPT{\_}RVM.1.1 ]
%[visit_definition]

The TSF shall ensure that TSP enforcement functions are invoked
and succeed before each function within the TSC is allowed to
proceed.

%[depart_definition]
%[depart_definition_list_item]
\end{description}


%___________________________________________________________________________



\minisec{FPT{\_}SEP.1 TSF domain separation}
\begin{description}
%[visit_definition_list_item]
\item[FPT{\_}SEP.1.1 ]
%[visit_definition]

The TSF shall maintain a security domain for its own execution that
protects it from interference and tampering by untrusted
subjects.

%[depart_definition]
%[depart_definition_list_item]
%[visit_definition_list_item]
\item[FPT{\_}SEP.1.2 ]
%[visit_definition]

The TSF shall enforce separation between the
security domains of subjects in the TSC.

%[depart_definition]
%[depart_definition_list_item]
\end{description}


%___________________________________________________________________________



\minisec{FPT{\_}STM.1 Reliable time stamps}
\begin{description}
%[visit_definition_list_item]
\item[FPT{\_}STM.1.1]
%[visit_definition]

The TSF shall be able to provide reliable time stamps for its own use.

%[depart_definition]
%[depart_definition_list_item]
\end{description}


%___________________________________________________________________________



\subsection{TOE security assurance requirements}

The Evaluation Assurance Level chosen for this Evaluation is EAL 1.

The following TOE assurance requirements drawn from CC Part 3 are valid:


\begin{longtable}[c]{lp{7cm}p{3cm}}
  \toprule
  Identification & Description & Direct dependencies\\
  \midrule \endhead

  \textbf{ACM} & Configuration management (CM) &  \\
  ACM{\_}CAP.1 & Version numbers & None \\

  \textbf{ADO} & Delivery and Operation &  \\
  ADO{\_}IGS.1 & Installation, generation and start-up & AGD{\_}ADM.1 \\
  
  \textbf{ADV} & Development &  \\
  ADV{\_}FSP.1 & Informal Functional specification & ADV{\_}RCR.1 \\

  ADV{\_}RCR.1 & Representation correspondence: Information correspondence
  demonstration & None \\ 

  \textbf{AGD} & Guidance documents &  \\
  AGD{\_}ADM.1 & Administrator guidance & ADV{\_}FSP.1 \\
  AGD{\_}USR.1 & User guidance & ADV{\_}FSP.1 \\
  \textbf{ATE} &  &  \\ 
  ATE{\_}IND.1 & Independent testing - conformance & ADV{\_}FSP.1 AGD{\_}ADM.1 AGD{\_}USR.1 \\


  \bottomrule
  \caption{TOE Assurance Requirements}
            
\end{longtable}


%___________________________________________________________________________



\section{Security requirements for the IT environment}

The following security requirements exist for the IT environment:

\begin{itemize}

  \item The operating system is Windows 2000, Windows XP or Linux. Either all
      known security patches must have been installed.

  \item The Python Version is 2.3.4 or a compatible Bugfix release.

  \item The ZODB storage is FileStorage or FileStorage through a ZEO server.

  \item The client software must support ``protected authentication feedback'',
  to at least not echo a user's credentials in plain text (FIA{\_}UAU.7).

  \item The TOE can only be accessed through a ``trusted path'' using secure
      proxies, such as an HTTPS proxy like Apache with SSL, or Pound. Users are
      taught to make correct use of secure channels (e.g. accepting only valid
      SSL certificates).

  \item If external IT systems are used, a trusted channel between the TOE and
      those systems must be provided by the TOE host environment.  For example,
      while the TOE may communicate with clients on a public network through a
      specific port allowed through a firewall, all communication with other IT
      systems should be over a (virtual) private network.

\end{itemize}
%___________________________________________________________________________

\chapter{TOE summary specification}


\section{TOE security functions}  \label{toe-sec-funcs}


The major functions implemented by the TOE are:


%___________________________________________________________________________


\subsection{Protection}

The protection subsystem is responsible for controlling the access of subjects
to objects.  It does this through the use of security proxies.  Any non-basic
objects that an interaction accesses is wrapped in a security proxy.  All
operations on these non-basic objects are performed through the security
proxies. Security proxies use the authorization system to decide whether access
is allowed.  Any non-basic results of operations performed through security
proxies are security proxied.

%___________________________________________________________________________



\subsection{Authentication}

Zope provides a flexible authentication schema that by default supports HTTP
Basic Auth and is extensible to support different data
storages. Zope defines interfaces to implement different mechanisms for
authentication data schemas as well as authentication mechanisms. By default
Zope provides components to store username/password pairs in the ZODB and to
authenticate with a username/password pair through HTTP Basic Authentication
and FTP authentication.


%___________________________________________________________________________



\subsection{Authorization / Access Control}

To determine whether an operation under a given subject is allowed, Zope has an
authorization subsystem (aka access control). The authorization subsystem uses
pluggable policies to allow the implementation of different rule sets. Zope
provides a default security policy called 'zopepolicy'.

Policies implement a method 'checkPermission' to determine whether the
requested access is allowed or not. Policies define the information required to
make authorization decisions.  Policies therefore can be implemented to extend
or reduce the current functionality, e.g. for introducing groups.

The default policy considers roles, grants and denials, location and principals
to drive the decision. Permissions can be granted or denied to principals
directly or by roles.  Subjects can consider multiple principals if the
execution of untrusted code is involved.

If no principal is associated with a subject, the subject is allowed to perform
any operation. The publisher component is required to bind an anonymous
principal with the special role ``zope.Anonymous'' if no user has been
authenticated.

Every principal is automatically granted the ``zope.Anonymous'' role which can't be
denied by any means. Also, every principal is granted the ``zope.Public''
permission which can't be denied by any means.

The 'zopepolicy' favors more specific declarations (permissions granted to
principals over permissions granted to roles or grants that are nearer in terms
of location) in contrast to more general (e.g. global declarations). Therefore
it is possible to make exceptions for individuals from permissions granted to a
role.


%___________________________________________________________________________


\subsection{Configuration}

\begin{itemize}
  \item assigning roles/permissions/etc via zcml
  \item assigning permissions/roles/users via management interface
  \item assigning permissions/roles/users via API
\end{itemize}

- only allow consistent configurations to be accepted

\subsection{Auditing}

Zope provides an auditing system that listens for events within Zope according
to the SFRs described above. It is implemented using the event framework of
Zope 3 to subscribe to the audit relevant events and log them appropriately.
The infrastructure provided (event listener + logger) satisfies the
requirements as described in FAU{\_}GEN.1 and FAU{\_}GEN.2.

Zope relies on the operating system to deliver reliable time stamps for the
audit log.


%___________________________________________________________________________



\subsection{Transaction management}

Most data is stored on persistent objects. The transaction machinery rolls back
all data that is stored on persistent objects.


%___________________________________________________________________________



\subsection{Undo}
\begin{itemize}
\item {} 
storage support

\item {} 
can redo

\item {} 
any level

\item {} 
can undo a transaction, if there aren't any un-undone transactions that touch
the same objects, if a conflict arises while changing the objects during
undo, application level conflict resolution is used to try to resolve a
conflict. if a conflict can't be resolved the undo can't be done.

\end{itemize}

Undo only allows the rollback of serial N last transaction to avoid integrity issues.


%___________________________________________________________________________



\subsection{Publication / Server}

The publisher allows users to communicate with the Zope server through a
network, using standard protocols and client software like HTTP and browsers or
FTP and FTP clients. The publisher is the only valid entry point to communicate
with a Zope 3 application as it also is the starting point for the protection
subsystem to be involved.

The publisher is extensible to allow support for further protocols.

To support FIA{\_}UAU.1 the implementation of a protocol includes the ability to
communicate with a user for requesting authentication data. The ability to
present credentials is specific to the used protocol and clients. By default
HTTP Basic Auth and FTP authentication are supported.

To support FIA{\_}USB.1, the publisher also returns the credentials to Zope and
calls the authentication subsystem to validate this data and binds the
authenticated principal to the running interaction.

%___________________________________________________________________________


\subsection{Automated Tests}

Zope provides a suite of automated tests that allow the user to ensure that the
security functionality implemented with a delivered package is consistent with
the tests. Those tests can be run in offline mode.


%___________________________________________________________________________



\subsection{Python Environment}

As Zope relies on Python and the host environment to provide reliable time
stamps. Changes to the external clock are not audited within the system as we
regard then as beeing out of scope.  Therefore external log mechanisms (Syslog
on Unix hosts or the Event log on Windows hosts) should be consulted to detect
those changes. (FPT{\_}STM.1)

%___________________________________________________________________________


\section{Assurance measures}


%___________________________________________________________________________



\subsection{AM{\_}ACM: CONFIGURATION MANAGEMENT}

A document describing the configuration management will be provided.


%___________________________________________________________________________



\subsection{AM{\_}ADO: DELIVERY AND OPERATION}

A document describing the delivery and operation of the TOE will be provided.


%___________________________________________________________________________


\subsection{AM{\_}ADV: DEVELOPMENT}

A functional specification,  a RCR document and a security policy model
(ADV\_SPM) will be provided.

%___________________________________________________________________________



\subsection{AM{\_}AGD: GUIDANCE DOCUMENTS}

The guidance documents AGD{\_}ADM and AGD{\_}USR will be provided.


%___________________________________________________________________________



\subsection{AM{\_}ATE: TESTS}

No deliverable. Only independend testing from the evaluator is needed.


%___________________________________________________________________________



\chapter{PP claims}

No PP compatibility is beeing claimed.


%___________________________________________________________________________


\chapter{Rationale}


%___________________________________________________________________________



\section{Security Objectives Rationale}

The following table shows that all threads and assumptions are covered
by a security objectives. 

  \begin{longtable}{rRRRRRRRRRRRRRRR}
    \toprule
              & T.IA  & T.Perm &T.Operation&T.AuditFake& T.RIP&T.Transaction&T.Undo &T.Timestamps & T.Host & A.OS & A.Admin & A.Network & A.Client & A.Credential  \\
    \midrule\endhead
O.IA         &  \oh  &       &            &            &      &             &       &             &        &      &       \\
O.Delegation &       &   \oh &            &            &      &             &       &             &        &      &        \\
O.Audit      & \oh   &   \oh &     \oh    &            &      &             &  \oh  &             &        &      &        \\
O.Protect    &       &       &            &    \oh     &      &             &       &             &        & \oh  &        \\
O.Access     &       &       &      \oh   &            &      &    \oh      &       &             &   \oh  &      &        \\
O.Integrity  &       &       &            &            &  \oh &             &       &             &        &      &        \\
O.Attributes &       &       &            &            &  \oh &             &  \oh  &             &        &      &        \\
O.ManageRisk &   \oh &       &            &            &      &             &       &             &        &      &        \\
\midrule
OE.OS        &       &       &            &            &      &             &       &    \oh      &        &  \oh &         &         &            &          \\
OE.Trust     &       &       &            &            &      &             &       &             &        &      &  \oh    &         &            &            \\
OE.Auditlog  &       &       &            &            &      &             &       &             &        & \oh  &         &         &            &                   \\  
OE.Network   &       &       &            &            &      &             &       &             &        &      &         &  \oh    &            &                   \\   
OE.Client    &       &       &            &            &      &             &       &             &        &      &         &         &    \oh     &                   \\
OE.Credential&       &       &            &            &      &             &       &             &        &      &         &         &            &    \oh    \\ 
\bottomrule
  \caption{Mapping of Threats and Assumptions to Security Objectives}
  \label{tab-SOR}
\end{longtable}

The following list explains why the objectives cover
the threads and assumptions.

\begin{description}
  
  \item[O.IA:] This security objective is necessary to counter the threat
  \textbf{T.IA} because it requires that users must be accurately identified
  and authenticated or incorporate the anonymous principal.

  \item[O.Delegation:] This security objective is necessary to counter the
  threat \textbf{T.Perm} because a user must only be able to delegate the permissions
  he is allowed to delegate. It must not be possible for him to gain any extra
  permissions.
  
  \item[O.Audit:] This security objective is necessary to detect and recover
  from most threats: \textbf{T.IA, T.Perm, T.Operation and T.Undo}
  as those events are logged by the audit log.
  
  \item[O.Protect:] This security objective is necessary to counter the threat
  \textbf{T.AuditFake} because it protects the audit data generation function
  and thereby prevents logging of false information. It also helps to covers
  the assumption \textbf{A.OS} because self-protection mechanisms help to
  dtect security problems in the runtime environment.
  
  \item[O.Access:] This security objective is necessary to counter the threat
  \textbf{T.Operation} because it prevents performing operations on an object
  without having the correct permission. It also counters the threat
  \textbf{T.Host} because functions are objects and thereby protected.

  O.Access also counters the threat \textbf{T.Transaction} because transaction
  managing functions are also objects and therefore protected.

  \item[O.Integrity:] This security objective is necessary to counter the
  threat \textbf{T.RIP} because it prevents that any data will be written if
  an unhandled error occurs.
  
  \item[O.Attributes:] This security objective is necessary to counter the
  threats \textbf{T.Undo} and \textbf{T.RIP} because it
  prevents an attacker form setting inconsistent security attributes from
  which he could gain more access than intended.

  \item[O.ManageRisk:] This security objective is necessary to counter the
  threat \textbf{T.IA} because it makes it less likely that an attacker
  impersonates a principal which allows operations with high negative impact
  since those principals are better protected.

  \item[OE.OS:] This security objective is necessary to both counter the
  threat \textbf{T.Timestamps} and cover the assumption \textbf{A.OS} because
  it asserts that the machine and the operating system the TOE is running on
  are physically secure. This means an attacker cannot access the machine
  directly, i.e. around Zope.

  \item[OE.Auditlog:] This security objective covers the assumption
  \textbf{A.OS}. To keep the operating system secure and detect possible
  intrusions it is vital to continuously monitor the audit log. It is also
  important to keep the Auditlog in a safe place to have enough information to
  recover from an attack.

  \item[OE.Network:] This security objective covers the assumptions
  \textbf{A.Network} because it asserts that all
  network !(!5
  compromising the integrity.

  \item[OE.Client:] This security objective covers the assumption
  \textbf{A.Client} because it makes sure that the identification and
  authentication data is not monitored or interfered.

  \item[OE.Credential:] This security objective covers the assumption
  \textbf{A.Credential} because it demands that the user is keeping the
  credentials to authenticate secret.
  
\end{description}
%___________________________________________________________________________



\section{Security requirements rationale}


\begin{longtable}{rRRRRRRRR}
        \toprule
                            & O.IA & O.Delegation & O.Audit & O.Protect & O.Access & O.Integrity & O.Attributes & O.ManageRisk \\
        \midrule\endhead

FAU\_GEN.1                  &      &              & \oh     &           &          &             &              &              \\
FAU\_GEN.2                  &      &              & \oh     &           &          &             &              &              \\
FDP\_ACC.2                  &      & \oh          &         &           & \oh      &             &              &              \\
FDP\_ACF.1                  &      &              &         &           &  \oh     &             &              &              \\
FDP\_RIP.1                  &      &              &         &           &          &             &  \oh         &              \\
FDP\_ROL.2\_Transactions    &      &              &         &           &          &   \oh       &              &              \\
FDP\_ROL.1\_Undo            &      &              &         &           &          &             &  \oh         &              \\
FIA\_AFL\_z.1               &      &              &         &   \oh     &          &             &              &              \\
FIA\_ATD.1                  & \oh  &  \oh         &   \oh   &           & \oh      &             &              &              \\
FIA\_UAU.1                  & \oh  &              &         &           &          &             &              &              \\
FIA\_UAU.5                  & \oh  &              &         &           &          &             &              & \oh          \\
FIA\_UAU.6                  & \oh  &              &         &           &          &             &              & \oh          \\
FIA\_UID.1                  & \oh  &              &         &           &          &             &              &              \\
FIA\_USB.1                  & \oh  &              &         &           &          &             &              &              \\
FMT\_MOF.1                  &      &              &         &  \oh      &          &             &              & \oh          \\
FMT\_MSA.1                  & \oh  &  \oh         &         &           &          &             &              &              \\
FMT\_MSA.2                  &      &              &         &           &          &             &  \oh         &              \\  
FMT\_MSA.3                  &      &              &         & \oh       &          &             &  \oh         &              \\
FMT\_SMR.1                  &      &              &         &           &          &             &              &              \\
FPT\_AMT.1                  &      &              &         & \oh       &          &             &              &              \\
FPT\_RVM.1                  &      &              &         &           &  \oh     &             &              &              \\
FPT\_FLS.1                  &      &              &         &  \oh      &          &   \oh       &              &              \\
FPT\_SEP.1                  &      &              &         &   \oh     &          &             &              &   \oh        \\
FPT\_STM.1                  &      &              &  \oh    &           &          &             &              &              \\
 \bottomrule
 \caption{Mapping of Security Objectives to Security Functional Requirements}
\end{longtable}

\subsection{SFR Component dependency analysis}

\begin{longtable}{rp{8cm}}
        \toprule
        SFR                 &   Depends on  \\
        \midrule\endhead
FAU\_GEN.1                  &   FPT\_STM.1   \\
FAU\_GEN.2                  &   FAU\_GEN.1, FIA\_UID.1 \\
FDP\_ACC.2                  &   FDP\_ACF.1 \\
FDP\_ACF.1                  &   FDP\_ACC.1, FMT\_MSA.3 \\
FDP\_RIP.1                  &   -- \\
FDP\_ROL.2\_Transactions    &   -- \\
FDP\_ROL.1\_Undo            &   -- \\
FIA\_AFL\_z.1               &   FIA\_UAU.1 \\
FIA\_ATD.1                  &   -- \\
FIA\_UAU.1                  &   FIA\_UID.1 \\
FIA\_UAU.5                  &   -- \\
FIA\_UAU.6                  &   -- \\
FIA\_UID.1                  &   -- \\
FIA\_USB.1                  &   FIA\_ATD.1 \\
FMT\_MOF.1                  &   FMT\_SMR.1 \\
FMT\_MSA.1                  &   FMT\_SMR.1 \\
FMT\_MSA.2                  &   ADV\_SPM.1, FMT\_MSA.1, FMT\_SMR.1 \\
FMT\_MSA.3                  &   FMT\_MSA.1, FMT\_SMR.1 \\
FMT\_SMR.1                  &   FIA\_UID.1 \\
FPT\_AMT.1                  &   -- \\
FPT\_RVM.1                  &   -- \\
FPT\_FLS.1                  &   ADV\_SPM.1 \\
FPT\_SEP.1                  &   -- \\
FPT\_STM.1                  &   -- \\
\bottomrule
   \caption{SFR Dependency Analysis}
\end{longtable}

All dependencies required by the chosen SFRs are covered. See table XXX.

\subsection{O.IA --- Identification and Authentication}

    A central part of the security machinery within the TOE is the correct
    identification and authentification of users.

    This is covered by the activities:

    \begin{description}
        \item[Asking for and validating a user's credentials:]

            The TOE holds information to uniquely identify a principal and its
            required credentials (FIA\_ATD.1).
            
            The TOE presents the user with a prompt to supply his credentials
            if an operation requires an identified and authenticated principal (FIA\_UAU.1, FIA\_UID.1)

            Depending on the communication channel, the system selects a
            suitable authentication mechanism to ask a user for his
            credentials. (FIA\_UAU.5)

            If an authenticated user does not have enough permission grants to
            perform an operation, he will get the chance to authenticate with
            other credentials. (FIA\_UAU.6)

            If the credentials stored at the user agent expire (e.g. cookies in
            a web browser), the user will be asked to represent his credentials
            before performing any further operation. (FIA\_UAU.6)

        \item[Binding users to the correct principals:]

            The TOE allows users to interact with the system without presenting
            credentials by binding unauthenticated users to the ``Anonymous''
            principal. This allows parts of applications to be accessible without
            presenting any credentials. (FIA\_UAU.1)

            Once a user has been identified and authenticated, the subject of
            the operation is bound to the user by selecting the correct
            principal. (FIA\_USB.1)

        \item[Managing required security attributes:]

            The TOE manages the required security attributes (permission grants
            and denials, credentials, \dots). Special permissions are required
            to read or write certain security attributes. (FMT\_MSA.1)

        \item[Associating principals with the correct security attributes:]

            This is covered by FIA\_ATD.1 and FIA\_USB.1

    \end{description}

\subsection{O.Delegation  --- Securely delegate control}

    Changing permission grants and denials allows the delegation of permission
    grants and denials to other users. Administrators that have grants for all
    permissions introduce new users to the system by delegating the required
    permissions to them (e.g. via roles, direct permission grants or denials).

    Delegating control is a normal operation performed on the TOEs objects. To
    grant a permission special meta permissions are introduced that control the
    ability to delegate a permission. (FMT\_ATD.1)

    Those operations are securely managed because they are covered by the TSF
    (FDP\_ACC.2) and follow special rules regarding the management of security
    attributes. (FMT\_MSA.1)

\subsection{O.Audit --- Provide a reliable security audit trail}

    The TOE shall provide functionality to generate audit data (FAU\_GEN.1,
    FAU\_GEN.2).

    The TOE includes reliable time stamps to guarantee reasonable data to be
    logged (FPT\_STM.1) and connects all events with the relevant user
    attributes. (FIA\_ATD.1)

\subsection{O.Protect --- Protect the TOE from external tampering}

    The TOE provides some effort to not allow an insecure situation that
    resulted from tampering with the system. Most situations have to be avoided
    due to correct appliance of the environmental requirements though.

    As the TOE is normally run with access through open communication channels
    like the internet, credentials very likely might be compromised by brute
    force attacks. This is avoided by applying FIA\_AFL\_z.1.

    Changing the behaviour of security functions is a critical operation.
    Therefore a set of well known permissions and roles are established to
    easily identify people that are able to change any security relevant
    behaviour. (FMT\_MOF.1)
    
    In the case of data loss, failure of subsystems or unexpected situations,
    the usage of FMT\_MSA.3 allows the system to stay in the most secure state
    possible. Asserting restrictive default values for security attributes
    avoids permission elevation and results in a better protected TOE.
    
    Using abstract machine tests, the system is able to check if the security
    code has been modified and does not hold to the assumptions of the security
    machinery anymore. (FPT\_AMT.1)

    By keeping a secure state (FPT\_FLS.1) the system is able to protect itself
    during (intentional or not) hardware failures or other environmental
    interruptions.
    
    The TOE holds a special domain for running untrusted codes that allows
    external entities not to directly modify or call any security relevant
    attributes or functions. (FPT\_SEP.1)

\subsection{O.Access --- Mediate every access to objects}

    Mediating every access to an object through operations is another major
    objective to enforce the TSP. (FDP\_ACC.2)

    A set of attributes and rules is used to describe how to apply those
    attributes for deriving an access decision. (FDP\_ACF.1, FIA\_ATD.1)  

    Certain special operations like import and export of user data are handled
    in a way that they cannot be exploited for exporting data a user doesn't
    have access to nor importing data that may extend a users privileges in a
    not intended way. 

    To ensure the non-bypassability of the TSP a special paradigm (security
    proxies) for accessing TOE objects from external entities. (FIA\_RVM.1)
    
\subsection{O.Integrity --- Ensure faultless data}

    Providing an ACID compatible transaction management system that allows
    secure rollback from a failed transaction satisfies the objective to have
    the system stay in an integer state. (FDP\_ROL.2\_Transactions, FPT\_FLS.1)

    The rollback is performed by the TOE automatically as soon as an error is
    encountered and not handled by any application logic.

\subsection{O.Attributes --- Ensure consistent security attributes}

    To assure an enduring consistent state of all security attributes we
    enforce the security policy model upon any changes to security attributes.
    (FMT\_MSA.2) Additionally static security attribute initialization assures
    a predictable and secure state if no specific attributes are given.
    (FMT\_MSA.3)

    Special functions like residual information protection (FDP\_RIP.1) and
    rollback to historic revisions (FDP\_ROL.1\_Undo) also have to assure that
    the used security attributes do not reference invalid identifiers. 
    
\subsection{O.ManageRisk --- Provide choice of flexibility versus security}

    To manage the risk of using stronger authentication schemes for sensible
    operations in opposition of weaker authentication schemes for less sensible
    operations, the TOE allows a selection of the authentication system to
    happen. (FIA\_UAU.5) To increase a users level of access for accessing more
    sensible operations, a reauthentication may be needed. (FIA\_UAU.6)
    
    The decision what authentication schemes are needed for which operations
    can be managed by authorised administrators. (FMT\_MOF.1)

    Additionally code can be run either within the trusted or untrusted
    security domains of the TOE. Installing code in the trusted security domain
    requires an external entity that has access to the physical secure host to
    install software into the TOE. This allows developers to trade off between
    functionality of their code and the level of trust they have to gain from
    administrators installing their extensions. FPT\_SEP.1 supports the
    distinction between the trusted and untrusted domain.

\section{Summary Specification Rationale}

\subsection{Security Functions Rationale}

\begin{longtable}{rRRRRRRRRRR}
        \toprule
                    & Protection & Authentication & Authorization & Auditing & Configuration & Transaction management & Undo & Publication/Server & Automated Tests & Python Environment \\
        \midrule\endhead
FAU\_GEN.1          &            &                &               & \oh      &               &                        &      &                    &                 &                    \\   
FAU\_GEN.2          &            &                &               & \oh      &               &                        &      &                    &                 &                    \\   
FDP\_ACC.2          & \oh        &                &               &          &               &                        &      & \oh                &                 &                    \\   
FDP\_ACF.1          &            &                &  \oh          &          &               &                        &      &                    &                 &                    \\   
FDP\_RIP.1          &            &                &               &          & \oh           &                        &      &                    &                 &                    \\   
FDP\_ROL.2\_TRANSACTIONS
                    & \oh        &                &  \oh          &          &               & \oh                    &      &                    &                 &                    \\   
FDP\_ROL.1\_UNDO    & \oh        &                &  \oh          &          &               &                        & \oh  &                    &                 &                    \\   
FIA\_AFL\_z.1       &            &  \oh           &               &          &               &                        &      & \oh                &                 &                    \\   
FIA\_ATD.1          &            &                &               &          & \oh           &                        &      &                    &                 &                    \\   
FIA\_UAU.1          &            &                &               &          &               &                        &      & \oh                &                 &                    \\   
FIA\_UAU.5          &            &  \oh           &               &          &               &                        &      & \oh                &                 &                    \\   
FIA\_UAU.6          &            &  \oh           &               &          &               &                        &      & \oh                &                 &                    \\   
FIA\_UID.1          &            &                &               &          &               &                        &      & \oh                &                 &                    \\   
FIA\_USB.1          &            &                &               &          &               &                        &      &  \oh               &                 &                    \\   
FMT\_MOF.1          & \oh        &  \oh           &  \oh          &          & \oh           &                        &      &                    &                 &                    \\   
FMT\_MSA.1          &            &                &  \oh          &          & \oh           &                        &      &                    &                 &                    \\   
FMT\_MSA.2          &            &                &               &          & \oh           &                        &      &                    &                 &                    \\   
FMT\_MSA.3          &            &                &  \oh          &          & \oh           &                        &      &                    &                 &                    \\   
FMT\_SMR.1          &            &                &  \oh          &          & \oh           &                        &      &                    &                 &                    \\   
FPT\_AMT.1          &            &                &               &          &               &                        &      &                    &    \oh          &                    \\   
FPT\_RVM.1          & \oh        &                &               &          &               &                        &      &  \oh               &                 &                    \\   
FPT\_FLS.1          &            &                &               &          &               &     \oh                &      &                    &                 &                    \\   
FPT\_SEP.1          &  \oh       &                &               &          &               &                        &      &                    &                 &                    \\ 
FPT\_STM.1          &            &                &               &          &               &                        &      &                    &                 &   \oh              \\       
    \bottomrule
    \caption{Security Functions Rationale} % XXX

\end{longtable}

\subsubsection{Suitability of SF to meet the SFRs}

\minisec{FDP\_ACC.2 --- Complete Access Control}

Complete access control is achieved by the \textbf{Protection} subsystem. The
\textbf{Publication} subsystem serves as a single entry point to the Zope 3
application which wraps all published objects into security proxies. The
transient nature of security proxies then covers that all subsequent accesses
are security proxied as well. Thereby all operations among those objects are
covered by the protection subsytem which enforces the formal security policy.

\minisec{FDP\_RIP.1 --- Subset residual information protection}

RIP is covered by the \textbf{Configuration} subsystem. This subsystem provides
an API that verifies that the removal of security attributes (like user
accounts) also results in a consistent removal of depending security attributes
(group memberships, role grants, permission grants \ldots).

\minisec{FDP\_ROL.2\_TRANSACTIONS --- Advanced Rollback}

The \textbf{Transaction management} of ZODB allows rollback of transaction. The
\textbf{Protection} and \textbf{Authorization} subsystems (in place by the
complete access control) will deny the unauthorized use of those management
functions. As a result only the subject of a transaction is able to roll back
it's corresponding transaction.

As transactions are only valid within a single subject (operation), there is no
possibility to cancel other transactions through the use of the
\textbf{Publication} subsystem.

\minisec{FDP\_ROL.1\_UNDO --- Basic Rollback}
(FIA{\_}UAU.7), 
The \textbf{Undo} subsystem covers undoing old transactions in a secure and
consistent manner. Old transactions that are not to be undone consistently
are not allowed to be undone.

The \textbf{Protection} and \textbf{Authorization} subsystems ensure that only
authorized principals can access this functionality. This is similar as to the
rollback of transactions.

\minisec{FIA\_AFL\_z.1 --- Authentication Failure Handling}

AFL is handled in cooperation of the \textbf{Authentication} and
\textbf{Publication} subsystem. The \textbf{Publication} subsystem identifies
individual authentication trials and uses the authentication subsystem to
update the security attributes that store the information about failed
requests. The split of this functionality is needed as the
\textbf{Authentication} subsystem may not be able to distinguish two requests
to be different user initiated requests or started off at another point in the
application.
  

\minisec{FIA\_UAU.1, FIA\_UID.1 --- Timing of authentication and identification}

The \textbf{Publication} subsystem detects provided credentials and existing
sessions on the implemented network protocols. It then either identifies and
authenticates the user for this subject or uses the anonymous principal to
perform the requested operation. If the anonymous principal is not allowed to
perform the requested operation, the \textbf{Publication} subsystem challenges
the user to provide sufficient credentials for authentication and
identification.

\minisec{FIA\_UAU.5 --- Multiple Authentication Systems}

The \textbf{Publication} and \textbf{Authentication} subsystems work together
to identify a meaningful way of asking a user for his credentials. 

The \textbf{Authentication} subsystem can then implement different schemes for
validating the credentials that the \textbf{Publication} system retrieved from
the user.

The choice of retrieval and verification can depend on the network protocol
(FTP, HTTP) and the strength of authentication that is requests (e.g. password,
client certificates).

\minisec{FIA\_UAU.6 --- Re-authentication}

If an operation could not be performed due to missing permission grants, the
\textbf{Publication} subsystem may -- instead of denying further operation --
may ask the user to provide other credentials to authenticate for a different
principal.

\emph{Note:} This is implemented by the same scheme that is used to initially
retrieve credentials from a user when the operation could not be performed by
the anonymous principal.

\minisec{FIA\_USB.1 --- User-Subject Binding}

When the \textbf{Publication} system sets up to perform an operation, it
creates a context of ``interaction''. This interaction is always associated
with (at least) one principal. If a principal was authenticated during the
creation of this context, the interaction gets associated with this principal.
Otherwise the anonymous principal will be bound to the subject.

Binding a principal to an interaction transitively associates the required security
attributes (e.g. permission grants) to this interaction.

\minisec{FMT\_MOF.1 --- Management of Security Functions}

Management of security functions happens by different physical ways (editing
configuration files, working with the web interface, writing scripts) that are
all addressing a single security configuration API that is offered by the
\textbf{Configuration}, \textbf{Authentication} and \textbf{Authorization}
subsystems. Access to those subsystems is -- as always -- covered by the
complete access control policy and enforced by the \textbf{Protection}
subsystem.

\minisec{FMT\_MSA.1 --- Management of Security Attributes}

The management of security attributes is provided by the \textbf{Configuration}
subsystem. This happens by using the same API as for FMT\_MOF.1 including the
different ways of accessing this API.

\minisec{FMT\_MSA.2 --- Secure Security Attributes}

The \textbf{Configuration} subsystems API for managing security functions and
attributes perform consistency checks upon the change of any security
attributes. This includes for example the check of dependencies that the
removal of principals also has the effect of removal of all dependent
permission grants and denials. 

Also only already existing identifiers (user names, permission names) may only
be used as references if they have been defined previously.

\minisec{FMT\_MSA.3 --- Static Attribute Initialization}

A set of fixed rules that are used whenever an attribute definition is missing
realize the static attribute initialisation. These rules are implemented in the
different subsystems (\textbf{Authorization} and \textbf{Configuration})
whenever a specific attribute would be used or defined.

\minisec{FMT\_SMR.1 --- Security roles}

The \textbf{Authorization} system resolves roles that users hold into
permissions they are granted or denied. The configuration system holds the
definition of what users posess which roles and which roles are granted which
permissions.

Pre-defined permission/role-mappings are delivered with the certified Zope
configuration to match the Administrator, Grantor and User roles. 

\minisec{FPT\_RVM.1 --- Non-bypassability of the TSP}

The concept of the \textbf{Protection} system is to put a layer of protection
around any object that is beeing accessed from an interaction. It is designed
in a transitive manner that it will not allow any computation to bypass it.

\minisec{FPT\_FLS.1 --- Failure With Preservation of Secure State}

The ZODB's \textbf{Transaction management} functions implement an ACID
compatible transaction scheme. The correct implementation of the storage layer
which has been tested extensively for FileStorage and ZEOStorage preserves
consistent and secure states on process termination and host restart. Resource
exhaustion will either result in a process termination (RAM usage) or denial of
service (upon disk usage) not allowing any further operation until the host
administrator resolves the situation.

\minisec{FPT\_SEP.1 --- TSF domain seperation}

The \textbf{Protection} subsystem allows code that has been brought to the
system via installation on the host filesystem to remove security proxies. This
results in the ability (for performance or functional reasons) to write code
that calls system functions and internal APIs without disturbing the protected
code areas.

When data is passed from the trusted domain into an untrusted (read: security
proxied domain), the Protection system will prevent the elevation of privileges
by putting reinstalling the layer of security proxies again.

\subsection{Assurance Measures}

The assurance measures are selected in accordance to EAL 1. Additionally due to
the selection of FMT\_MSA.2 the document ADV\_SPM has been selected.

%___________________________________________________________________________

\subsection{Choice of TOE security assurance requirements}

The choice of assurance requirements is based on the analysis of the security
objectives for the TOE and on functional requirements defined to meet these
objectives.

The assurance level is \textbf{EAL 1}.

%___________________________________________________________________________


\section{Evaluation Assurance Level rationale}

The Zope development community recognizes the need of mature and well defined
security functions by its users.

Therefore to meet this requirement the decision for an entry level evaluation
was made on the basis of resource constraints of available developers and
budget.

Additionally an entry level evaluation gives a glance to the community how a
certification may effect Zope's degree of documentation and stabilize the good
security history even more. Eventually this raises interest in Zope 3 for
projects that have strong requirements in respect to security and do seek free
alternatives to closed source projects.

It is intended to show that mature open source projects can outperform
proprietary systems not only on pure functional and monetary aspects but also
in domains that are typically governed by proprietary systems. Performing a
well known standardized evaluation also substantiates confidence and trust that
Zope as a free software project receives by it's users.

%___________________________________________________________________________



\chapter{Glossary}

\begin{description}

  \item[CC] Common Criteria (referenced as {[}CC])
  \item[SF] Security Function
  \item[SFP] Security Function Policy
  \item[SFR] Security Functional Requirement
  \item[ST] Security Targets
  \item[TOE] Target of Evaluation
  \item[TSF] TOE Security Functions

\end{description} 

%___________________________________________________________________________


\end{document}
